\documentclass[12pt,a4paper]{article}\usepackage[]{graphicx}\usepackage[]{color}
%% maxwidth is the original width if it is less than linewidth
%% otherwise use linewidth (to make sure the graphics do not exceed the margin)
\makeatletter
\def\maxwidth{ %
  \ifdim\Gin@nat@width>\linewidth
    \linewidth
  \else
    \Gin@nat@width
  \fi
}
\makeatother

\definecolor{fgcolor}{rgb}{0.345, 0.345, 0.345}
\newcommand{\hlnum}[1]{\textcolor[rgb]{0.686,0.059,0.569}{#1}}%
\newcommand{\hlstr}[1]{\textcolor[rgb]{0.192,0.494,0.8}{#1}}%
\newcommand{\hlcom}[1]{\textcolor[rgb]{0.678,0.584,0.686}{\textit{#1}}}%
\newcommand{\hlopt}[1]{\textcolor[rgb]{0,0,0}{#1}}%
\newcommand{\hlstd}[1]{\textcolor[rgb]{0.345,0.345,0.345}{#1}}%
\newcommand{\hlkwa}[1]{\textcolor[rgb]{0.161,0.373,0.58}{\textbf{#1}}}%
\newcommand{\hlkwb}[1]{\textcolor[rgb]{0.69,0.353,0.396}{#1}}%
\newcommand{\hlkwc}[1]{\textcolor[rgb]{0.333,0.667,0.333}{#1}}%
\newcommand{\hlkwd}[1]{\textcolor[rgb]{0.737,0.353,0.396}{\textbf{#1}}}%

\usepackage{framed}
\makeatletter
\newenvironment{kframe}{%
 \def\at@end@of@kframe{}%
 \ifinner\ifhmode%
  \def\at@end@of@kframe{\end{minipage}}%
  \begin{minipage}{\columnwidth}%
 \fi\fi%
 \def\FrameCommand##1{\hskip\@totalleftmargin \hskip-\fboxsep
 \colorbox{shadecolor}{##1}\hskip-\fboxsep
     % There is no \\@totalrightmargin, so:
     \hskip-\linewidth \hskip-\@totalleftmargin \hskip\columnwidth}%
 \MakeFramed {\advance\hsize-\width
   \@totalleftmargin\z@ \linewidth\hsize
   \@setminipage}}%
 {\par\unskip\endMakeFramed%
 \at@end@of@kframe}
\makeatother

\definecolor{shadecolor}{rgb}{.97, .97, .97}
\definecolor{messagecolor}{rgb}{0, 0, 0}
\definecolor{warningcolor}{rgb}{1, 0, 1}
\definecolor{errorcolor}{rgb}{1, 0, 0}
\newenvironment{knitrout}{}{} % an empty environment to be redefined in TeX

\usepackage{alltt}
\usepackage[utf8]{inputenc}
\usepackage{fullpage} % for margins
\usepackage{setspace} % for line spacing
\onehalfspacing % one and a half lines' spacing
\usepackage{amsmath}
\usepackage{amsfonts}
\usepackage{amssymb}
\author{Josh Nightingale}
\IfFileExists{upquote.sty}{\usepackage{upquote}}{}
\begin{document}

\section*{Results}







\begin{knitrout}
\definecolor{shadecolor}{rgb}{0.969, 0.969, 0.969}\color{fgcolor}\begin{figure}[t]

{\centering \includegraphics[width=0.7\textwidth,height=0.7\textwidth]{figure/distpairs-1} 

}

\caption[Correlation between the behaviour, diet, morphlogy and mean distance matrices]{Correlation between the behaviour, diet, morphlogy and mean distance matrices. Lower panels show scatter plots; upper panels show Pearson correlation coefficients, scaled according to their value.}\label{fig:distpairs}
\end{figure}


\end{knitrout}

In total, 106 surveys and 51 species were included in the analysis. 

The three distance sub-matrices (behaviour, diet and morphology) were poorly correlated with each other ($r \approx 0.1$), suggesting that these variables captured different aspects of species' functional ecology. 
However, the overall distance matrix used in subsequent analysis, calculated from the mean of the three sub-matrices, had a Pearson correlation $r \geq 0.5$ with all three sub-matrices, suggesting that all of these aspects were represented in the matrix used for analysis (Figure \ref{fig:distpairs}).

\clearpage
\subsection{Observed functional diversity}





In general, functional diversity appeared similar between habitats (Figure \ref{fig:fdsea}). Formal hypothesis testing confirmed this similarity. Only functional evenness varied between habitats (Table \ref{fdhabt}). Pairwise Mann-Whitney $U$ tests with the Benjamini-Hochberg (1995) correction showed that beach communities were less functionally even than grassland ($p = 0.001$) and lake ($p = \ensuremath{10^{-4}}$) communities, whereas there was no significant difference between grassland and lakes ($p = 0.83$).

% latex table generated in R 3.3.0 by xtable 1.8-2 package
% Tue May 10 03:17:06 2016
\begin{table}[t]
\centering
\caption{FDR-adjusted p values from Kruskal-Wallis tests comparing each functional diversity metric between beach, grassland and lake survey locations.} 
\label{fdhabt}
\begin{tabular}{rlr}
  \hline
 & Metric & p value \\ 
  \hline
1 & Dispersion & 0.409 \\ 
  2 & Divergence & 0.285 \\ 
  3 & Evenness & 0.001 \\ 
  4 & No.Species & 0.357 \\ 
  5 & Richness & 0.001 \\ 
   \hline
\end{tabular}
\end{table}


There was no apparent difference in any functional diversity metric between seasons (Figure \ref{fig:fdsea}), which was confirmed by formal hypothesis testing (Table \ref{fdseat}): there was no significant difference in any metric between summer and winter (FDR-adjusted Mann-Whitney $U$ tests: all $p > 0.05$).

\begin{knitrout}
\definecolor{shadecolor}{rgb}{0.969, 0.969, 0.969}\color{fgcolor}\begin{figure}[b]

{\centering \includegraphics[width=\textwidth,height=\textwidth]{figure/fdsea-1} 

}

\caption[Functional diversity metrics according to habitat (left) and season (right)]{Functional diversity metrics according to habitat (left) and season (right). Note that the $y$-axis has a different scale in each plot facet. Only functional evenness by habitat showed a significant difference, with beach communities significantly less even than the others (see text for statistics).}\label{fig:fdsea}
\end{figure}


\end{knitrout}

% latex table generated in R 3.3.0 by xtable 1.8-2 package
% Tue May 10 03:17:10 2016
\begin{table}[hb]
\centering
\caption{FDR-adjusted p values from Mann-Whitney tests comparing each functional diversity metric between summer and winter surveys.} 
\label{fdseat}
\begin{tabular}{rlr}
  \hline
 & Metric & p value \\ 
  \hline
1 & Dispersion & 0.69 \\ 
  2 & Divergence & 0.69 \\ 
  3 & Evenness & 0.69 \\ 
  4 & No.Species & 0.85 \\ 
  5 & Richness & 0.69 \\ 
   \hline
\end{tabular}
\end{table}


\begin{knitrout}
\definecolor{shadecolor}{rgb}{0.969, 0.969, 0.969}\color{fgcolor}\begin{figure}[h]

{\centering \includegraphics[width=\textwidth,height=\textwidth]{figure/fdhabsea-1} 

}

\caption[Habitat differences in functional diversity within each survey season]{Habitat differences in functional diversity within each survey season. No significant interaction between habitat and season for any functional diversity metric were revealed by two-way ANOVA models.}\label{fig:fdhabsea}
\end{figure}


\end{knitrout}



The pattern of functional divergence across habitat did not appear to vary seasonally (Figure \ref{fig:fdhabsea}). None of the two-way ANOVA models showed a significant interaction between habitat and season.

\clearpage
\subsection{Comparison with null model}
 
Species number and functional dispersion had SESs less than zero, while functional richness had an SES greater than 0 (Figure \ref{fig:SEShists}). Divergence and evenness were spread either side of 0. Significance of these scores can be seen in Table \ref{SESrestab}.

\begin{knitrout}
\definecolor{shadecolor}{rgb}{0.969, 0.969, 0.969}\color{fgcolor}\begin{figure}[h]

{\centering \includegraphics[width=\textwidth,height=0.7\textwidth]{figure/SEShists-1} 

}

\caption[Histograms of standardised effect size (SES) for each functional diversity metric, calculated from 1000 randomly simulated communities]{Histograms of standardised effect size (SES) for each functional diversity metric, calculated from 1000 randomly simulated communities}\label{fig:SEShists}
\end{figure}


\end{knitrout}

% latex table generated in R 3.3.0 by xtable 1.8-2 package
% Tue May 10 03:17:15 2016
\begin{table}[ht]
\centering
\caption{Median SES score and its associated p-value for each functional diversity metric} 
\label{SESrestab}
\begin{tabular}{rlrr}
  \hline
 & Metric & Median & p value \\ 
  \hline
1 & Dispersion & -15.7 & 0.057 \\ 
  2 & Divergence & -0.8 & 0.811 \\ 
  3 & Evenness & -2.8 & 0.623 \\ 
  4 & No.Species & -8.2 & 0.151 \\ 
  5 & Richness & 1.9 & 0.283 \\ 
   \hline
\end{tabular}
\end{table}


\clearpage
\subsection{Predicting community composition and function}

This bit first?

Species composition is significantly affected by both habitat and season; functional composition is affected only by habitat.

Analysing the subsets of trait data, there are seasonal differences in behaviour, but not in morphology nor in diet. How does this vary within habitats? Same food available, but need different method to acquire?

\clearpage
\subsection{Drill into CWMs?}

\end{document}
