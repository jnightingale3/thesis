\documentclass[12pt,a4paper]{article}\usepackage[]{graphicx}\usepackage[]{color}
%% maxwidth is the original width if it is less than linewidth
%% otherwise use linewidth (to make sure the graphics do not exceed the margin)
\makeatletter
\def\maxwidth{ %
  \ifdim\Gin@nat@width>\linewidth
    \linewidth
  \else
    \Gin@nat@width
  \fi
}
\makeatother

\definecolor{fgcolor}{rgb}{0.345, 0.345, 0.345}
\newcommand{\hlnum}[1]{\textcolor[rgb]{0.686,0.059,0.569}{#1}}%
\newcommand{\hlstr}[1]{\textcolor[rgb]{0.192,0.494,0.8}{#1}}%
\newcommand{\hlcom}[1]{\textcolor[rgb]{0.678,0.584,0.686}{\textit{#1}}}%
\newcommand{\hlopt}[1]{\textcolor[rgb]{0,0,0}{#1}}%
\newcommand{\hlstd}[1]{\textcolor[rgb]{0.345,0.345,0.345}{#1}}%
\newcommand{\hlkwa}[1]{\textcolor[rgb]{0.161,0.373,0.58}{\textbf{#1}}}%
\newcommand{\hlkwb}[1]{\textcolor[rgb]{0.69,0.353,0.396}{#1}}%
\newcommand{\hlkwc}[1]{\textcolor[rgb]{0.333,0.667,0.333}{#1}}%
\newcommand{\hlkwd}[1]{\textcolor[rgb]{0.737,0.353,0.396}{\textbf{#1}}}%

\usepackage{framed}
\makeatletter
\newenvironment{kframe}{%
 \def\at@end@of@kframe{}%
 \ifinner\ifhmode%
  \def\at@end@of@kframe{\end{minipage}}%
  \begin{minipage}{\columnwidth}%
 \fi\fi%
 \def\FrameCommand##1{\hskip\@totalleftmargin \hskip-\fboxsep
 \colorbox{shadecolor}{##1}\hskip-\fboxsep
     % There is no \\@totalrightmargin, so:
     \hskip-\linewidth \hskip-\@totalleftmargin \hskip\columnwidth}%
 \MakeFramed {\advance\hsize-\width
   \@totalleftmargin\z@ \linewidth\hsize
   \@setminipage}}%
 {\par\unskip\endMakeFramed%
 \at@end@of@kframe}
\makeatother

\definecolor{shadecolor}{rgb}{.97, .97, .97}
\definecolor{messagecolor}{rgb}{0, 0, 0}
\definecolor{warningcolor}{rgb}{1, 0, 1}
\definecolor{errorcolor}{rgb}{1, 0, 0}
\newenvironment{knitrout}{}{} % an empty environment to be redefined in TeX

\usepackage{alltt}
\usepackage[utf8]{inputenc}
\usepackage{fullpage} % for margins
\usepackage{setspace} % for line spacing
\onehalfspacing % one and a half lines' spacing
\usepackage{dcolumn} 
\usepackage{amsmath}
\usepackage{amsfonts}
\usepackage{amssymb}
\author{Josh Nightingale}
\IfFileExists{upquote.sty}{\usepackage{upquote}}{}
\begin{document}

\section*{Results}







\begin{knitrout}
\definecolor{shadecolor}{rgb}{0.969, 0.969, 0.969}\color{fgcolor}\begin{figure}[t]

{\centering \includegraphics[width=0.7\textwidth,height=0.7\textwidth]{figure/distpairs-1} 

}

\caption[Correlation matrix between the behaviour, diet, morphlogy and mean distance matrices]{Correlation matrix between the behaviour, diet, morphlogy and mean distance matrices. Lower panels show scatter plots; upper panels show Pearson correlation coefficients, scaled according to their value.}\label{fig:distpairs}
\end{figure}


\end{knitrout}

In total, 106 surveys and 51 species were included in the analysis. 

The three distance sub-matrices (behaviour, diet and morphology) were poorly correlated with each other ($r \approx 0.1$), suggesting that these variables captured different aspects of species' functional ecology. 
However, the overall distance matrix used in subsequent analysis, calculated from the mean of the three sub-matrices, had a Pearson correlation $r \geq 0.5$ with all three sub-matrices, suggesting that all of these aspects were represented in the matrix used for analysis (Figure \ref{fig:distpairs}).

\clearpage
\subsection{Observed functional diversity}





In general, functional diversity appeared similar between habitats (Figure \ref{fig:fdsea}). Formal hypothesis testing confirmed this similarity. Only functional evenness varied between habitats (Table \ref{fdhabt}). Pairwise Mann-Whitney $U$ tests with the Benjamini-Hochberg (1995) correction showed that beach communities were less functionally even than grassland ($p = 0.001$) and lake ($p = \ensuremath{10^{-4}}$) communities, whereas there was no significant difference between grassland and lakes ($p = 0.83$).


% Table created by stargazer v.5.2 by Marek Hlavac, Harvard University. E-mail: hlavac at fas.harvard.edu
% Date and time: Wed, May 11, 2016 - 00:13:44
\begin{table}[!htbp] \centering 
  \caption{FDR-adjusted p values from Kruskal-Wallis tests comparing each functional diversity metric between beach, grassland and lake survey locations} 
  \label{fdhabt} 
\small 
\begin{tabular}{@{\extracolsep{5pt}} cc} 
\\[-1.8ex]\hline 
\hline \\[-1.8ex] 
Metric & p value \\ 
\hline \\[-1.8ex] 
Dispersion & $0.409$ \\ 
Divergence & $0.285$ \\ 
Evenness & $0.001$ \\ 
No.Species & $0.357$ \\ 
Richness & $0.001$ \\ 
\hline \\[-1.8ex] 
\end{tabular} 
\end{table} 


There was no apparent difference in any functional diversity metric between seasons (Figure \ref{fig:fdsea}), which was confirmed by formal hypothesis testing (Table \ref{fdseat}): there was no significant difference in any metric between summer and winter (FDR-adjusted Mann-Whitney $U$ tests: all $p > 0.05$).

\begin{knitrout}
\definecolor{shadecolor}{rgb}{0.969, 0.969, 0.969}\color{fgcolor}\begin{figure}[b]

{\centering \includegraphics[width=\textwidth,height=\textwidth]{figure/fdsea-1} 

}

\caption[Functional diversity metrics according to habitat (left) and season (right)]{Functional diversity metrics according to habitat (left) and season (right). Note that the $y$-axis has a different scale in each plot facet. Only functional evenness by habitat showed a significant difference, with beach communities significantly less even than the others (see text for statistics).}\label{fig:fdsea}
\end{figure}


\end{knitrout}


% Table created by stargazer v.5.2 by Marek Hlavac, Harvard University. E-mail: hlavac at fas.harvard.edu
% Date and time: Wed, May 11, 2016 - 00:13:48
\begin{table}[!htbp] \centering 
  \caption{FDR-adjusted p values from Mann-Whitney tests comparing each functional diversity metric between summer and winter surveys} 
  \label{fdseat} 
\small 
\begin{tabular}{@{\extracolsep{5pt}} cc} 
\\[-1.8ex]\hline 
\hline \\[-1.8ex] 
Metric & p value \\ 
\hline \\[-1.8ex] 
Dispersion & $0.69$ \\ 
Divergence & $0.69$ \\ 
Evenness & $0.69$ \\ 
No.Species & $0.85$ \\ 
Richness & $0.69$ \\ 
\hline \\[-1.8ex] 
\end{tabular} 
\end{table} 


\begin{knitrout}
\definecolor{shadecolor}{rgb}{0.969, 0.969, 0.969}\color{fgcolor}\begin{figure}[h]

{\centering \includegraphics[width=\textwidth,height=\textwidth]{figure/fdhabsea-1} 

}

\caption[Habitat differences in functional diversity within each survey season]{Habitat differences in functional diversity within each survey season. No significant interaction between habitat and season for any functional diversity metric were revealed by two-way ANOVA models.}\label{fig:fdhabsea}
\end{figure}


\end{knitrout}



The pattern of functional divergence across habitat did not appear to vary seasonally (Figure \ref{fig:fdhabsea}). None of the two-way ANOVA models showed a significant interaction between habitat and season.

\clearpage
\subsection{Comparison with null model}
 
Species number and functional dispersion had SESs mostly lower than zero, while functional richness had SES usually greater than 0. Divergence and evenness were spread either side of 0, with medians slightly negative (Figure \ref{fig:SEShists}). Significance of these scores can be seen in Table \ref{SESrestab}.

\begin{knitrout}
\definecolor{shadecolor}{rgb}{0.969, 0.969, 0.969}\color{fgcolor}\begin{figure}[h]

{\centering \includegraphics[width=\textwidth,height=0.7\textwidth]{figure/SEShists-1} 

}

\caption[Histograms of standardised effect size (SES) for each functional diversity metric, calculated from 1000 randomly simulated communities]{Histograms of standardised effect size (SES) for each functional diversity metric, calculated from 1000 randomly simulated communities}\label{fig:SEShists}
\end{figure}


\end{knitrout}


% Table created by stargazer v.5.2 by Marek Hlavac, Harvard University. E-mail: hlavac at fas.harvard.edu
% Date and time: Wed, May 11, 2016 - 00:13:56
\begin{table}[!htbp] \centering 
  \caption{Median SES score and its associated p-value for each functional diversity metric} 
  \label{SESrestab} 
\small 
\begin{tabular}{@{\extracolsep{5pt}} ccc} 
\\[-1.8ex]\hline 
\hline \\[-1.8ex] 
Metric & Median & p value \\ 
\hline \\[-1.8ex] 
Dispersion & $$-$16.2$ & \textless 0.001 \\ 
Divergence & $$-$0.6$ & 0.012 \\ 
Evenness & $$-$2.5$ & 0.001 \\ 
No.Species & $$-$8.4$ & \textless 0.001 \\ 
Richness & $0.4$ & \textless 0.001 \\ 
\hline \\[-1.8ex] 
\end{tabular} 
\end{table} 


\clearpage
\subsection{Predicting community composition and function}


% Table created by stargazer v.5.2 by Marek Hlavac, Harvard University. E-mail: hlavac at fas.harvard.edu
% Date and time: Wed, May 11, 2016 - 00:13:57
% Requires LaTeX packages: dcolumn 
\begin{table}[tb] \centering 
  \caption{Predicting inter-survey dissimilarity of species composition with three regression models using distance matrices of (1) Survey habitat, (2) Season of survey and (3) Habitat and season} 
  \label{comp_pred_sg} 
\small 
\begin{tabular}{@{\extracolsep{5pt}}lD{.}{.}{-2} D{.}{.}{-2} D{.}{.}{-2} } 
\\[-1.8ex]\hline 
\hline \\[-1.8ex] 
 & \multicolumn{3}{c}{\textit{Dependent variable:}} \\ 
\cline{2-4} 
\\[-1.8ex] & \multicolumn{3}{c}{Species composition} \\ 
\\[-1.8ex] & \multicolumn{1}{c}{(1)} & \multicolumn{1}{c}{(2)} & \multicolumn{1}{c}{(3)}\\ 
\hline \\[-1.8ex] 
 Habitat & 0.17^{***} &  &  \\ 
  & \multicolumn{1}{c}{(0.15$, $0.18)} &  &  \\ 
  & & & \\ 
 Season &  & 0.01^{*} &  \\ 
  &  & \multicolumn{1}{c}{(0.001$, $0.02)} &  \\ 
  & & & \\ 
 Habitat and season &  &  & 0.18^{***} \\ 
  &  &  & \multicolumn{1}{c}{(0.16$, $0.19)} \\ 
  & & & \\ 
 Intercept & 1.05^{***} & 1.15^{***} & 1.06^{***} \\ 
  & \multicolumn{1}{c}{(1.05$, $1.06)} & \multicolumn{1}{c}{(1.14$, $1.16)} & \multicolumn{1}{c}{(1.05$, $1.07)} \\ 
  & & & \\ 
\hline \\[-1.8ex] 
Observations & \multicolumn{1}{c}{5,565} & \multicolumn{1}{c}{5,565} & \multicolumn{1}{c}{5,565} \\ 
R$^{2}$ & \multicolumn{1}{c}{0.13} & \multicolumn{1}{c}{0.001} & \multicolumn{1}{c}{0.07} \\ 
F Statistic (df = 1; 5563) & \multicolumn{1}{c}{830.33$^{***}$} & \multicolumn{1}{c}{4.24$^{*}$} & \multicolumn{1}{c}{447.20$^{***}$} \\ 
\hline 
\hline \\[-1.8ex] 
\textit{Note:}  & \multicolumn{3}{r}{$^{*}$p$<$0.05; $^{**}$p$<$0.01; $^{***}$p$<$0.001} \\ 
\end{tabular} 
\end{table} 


Species assemblages were highly significantly more dissimilar between sites in different habitats ($p < 0.001$), with this relationship explaining 13\% of variation. In addition, season was a significant predictor of community dissimilarity, though with very weak expanatory power ($R^2 = 0.001$). 

The distance matrix including both habitat and season was also a very significant predictor of species dissimilarity ($p < 0.001$), with intermediate $F$-statistic and $R^2$.


% Table created by stargazer v.5.2 by Marek Hlavac, Harvard University. E-mail: hlavac at fas.harvard.edu
% Date and time: Wed, May 11, 2016 - 00:13:58
% Requires LaTeX packages: dcolumn 
\begin{table}[!htbp] \centering 
  \caption{Predicting inter-survey dissimilarity of functional composition (Community Weighted Means) with three regression models using distance matrices of (1) Survey habitat, (2) Season of survey and (3) Habitat and season} 
  \label{func_pred_sg} 
\small 
\begin{tabular}{@{\extracolsep{5pt}}lD{.}{.}{-2} D{.}{.}{-2} D{.}{.}{-2} } 
\\[-1.8ex]\hline 
\hline \\[-1.8ex] 
 & \multicolumn{3}{c}{\textit{Dependent variable:}} \\ 
\cline{2-4} 
\\[-1.8ex] & \multicolumn{3}{c}{Functional composition} \\ 
\\[-1.8ex] & \multicolumn{1}{c}{(1)} & \multicolumn{1}{c}{(2)} & \multicolumn{1}{c}{(3)}\\ 
\hline \\[-1.8ex] 
 Habitat & 0.04^{***} &  &  \\ 
  & \multicolumn{1}{c}{(0.04$, $0.05)} &  &  \\ 
  & & & \\ 
 Season &  & -0.0004 &  \\ 
  &  & \multicolumn{1}{c}{(-0.01$, $0.004)} &  \\ 
  & & & \\ 
 Habitat and season &  &  & 0.04^{***} \\ 
  &  &  & \multicolumn{1}{c}{(0.03$, $0.05)} \\ 
  & & & \\ 
 Intercept & 0.21^{***} & 0.24^{***} & 0.22^{***} \\ 
  & \multicolumn{1}{c}{(0.21$, $0.22)} & \multicolumn{1}{c}{(0.23$, $0.24)} & \multicolumn{1}{c}{(0.21$, $0.22)} \\ 
  & & & \\ 
\hline \\[-1.8ex] 
Observations & \multicolumn{1}{c}{5,565} & \multicolumn{1}{c}{5,565} & \multicolumn{1}{c}{5,565} \\ 
R$^{2}$ & \multicolumn{1}{c}{0.05} & \multicolumn{1}{c}{0.0000} & \multicolumn{1}{c}{0.02} \\ 
F Statistic (df = 1; 5563) & \multicolumn{1}{c}{271.65$^{***}$} & \multicolumn{1}{c}{0.02} & \multicolumn{1}{c}{127.97$^{***}$} \\ 
\hline 
\hline \\[-1.8ex] 
\textit{Note:}  & \multicolumn{3}{r}{$^{*}$p$<$0.05; $^{**}$p$<$0.01; $^{***}$p$<$0.001} \\ 
\end{tabular} 
\end{table} 


Differences in community functional composition were also strongly related to habitat differences ($p < 0.001$), though with much less variation explained by this relationship ($R^2 = 0.047$; Table \ref{func_pred_sg}). 

Unlike species composition, functional composition did not vary between seasons. Attempts to model this relationship resulted in an $F$-statistic and $R^2$ very close to 0. The matrix including habitat and season was a significant predictor of species dissimilarity ($p < 0.001$), but with lower $F$-statistic and $R^2$ than the habitat-only model (Table \ref{func_pred_sg}).


% Table created by stargazer v.5.2 by Marek Hlavac, Harvard University. E-mail: hlavac at fas.harvard.edu
% Date and time: Wed, May 11, 2016 - 00:13:58
% Requires LaTeX packages: dcolumn 
\begin{table}[!htbp] \centering 
  \caption{functional subgroups} 
  \label{func_subs_pred_sg} 
\small 
\begin{tabular}{@{\extracolsep{5pt}}lD{.}{.}{-1} D{.}{.}{-1} D{.}{.}{-1} D{.}{.}{-1} D{.}{.}{-1} D{.}{.}{-1} } 
\\[-1.8ex]\hline 
\hline \\[-1.8ex] 
 & \multicolumn{6}{c}{\textit{Dependent variable:}} \\ 
\cline{2-7} 
\\[-1.8ex] & \multicolumn{2}{c}{Behaviour} & \multicolumn{2}{c}{Diet} & \multicolumn{2}{c}{Morphology} \\ 
\\[-1.8ex] & \multicolumn{1}{c}{(1)} & \multicolumn{1}{c}{(2)} & \multicolumn{1}{c}{(3)} & \multicolumn{1}{c}{(4)} & \multicolumn{1}{c}{(5)} & \multicolumn{1}{c}{(6)}\\ 
\hline \\[-1.8ex] 
 Habitat & 0.04^{***} &  & 0.1^{***} &  & 0.02^{***} &  \\ 
  Season &  & 0.001 &  & -0.001 &  & -0.001 \\ 
  Intercept & 0.2^{***} & 0.2^{***} & 0.2^{***} & 0.3^{***} & 0.2^{***} & 0.2^{***} \\ 
 \hline \\[-1.8ex] 
Observations & \multicolumn{1}{c}{5,565} & \multicolumn{1}{c}{5,565} & \multicolumn{1}{c}{5,565} & \multicolumn{1}{c}{5,565} & \multicolumn{1}{c}{5,565} & \multicolumn{1}{c}{5,565} \\ 
R$^{2}$ & \multicolumn{1}{c}{0.1} & \multicolumn{1}{c}{0.000} & \multicolumn{1}{c}{0.1} & \multicolumn{1}{c}{0.000} & \multicolumn{1}{c}{0.003} & \multicolumn{1}{c}{0.000} \\ 
F Statistic & \multicolumn{1}{c}{294.2$^{***}$} & \multicolumn{1}{c}{0.2} & \multicolumn{1}{c}{441.3$^{***}$} & \multicolumn{1}{c}{0.2} & \multicolumn{1}{c}{17.6$^{***}$} & \multicolumn{1}{c}{0.1} \\ 
\hline 
\hline \\[-1.8ex] 
\textit{Note:}  & \multicolumn{6}{r}{$^{*}$p$<$0.05; $^{**}$p$<$0.01; $^{***}$p$<$0.001} \\ 
\end{tabular} 
\end{table} 


Considering each category of functional traits separately confirmed that birds' behaviour, diet and morphology varies significantly between habitats, but not between seasons (Table \ref{func_subs_pred_sg}).


\clearpage
\subsection{Drill into CWMs? Differences amongst habitats in traits supported}

To do.

\end{document}
