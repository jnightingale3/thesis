\documentclass[12pt,a4paper,draft]{book}
\usepackage[utf8]{inputenc}
\usepackage{amsmath}
\usepackage{fullpage} % for margins
\usepackage{setspace} % for line spacing
\onehalfspacing % one and a half lines' spacing
\usepackage{amsfonts}
%\usepackage{fontspec,xunicode} %to control fonts
%\usepackage[T1]{fontenc} % for the Times font + math mode
\usepackage{amssymb}
\usepackage[backend=bibtex, style=authoryear]{biblatex}
\addbibresource{../../../../../Documents/library}
\author{Josh Nightingale}
\title{Foraging ecology of waterbirds on the north coast of Rio Grande do Sul}
\begin{document}

\maketitle

\tableofcontents

\chapter{Functional ecology}

\section{Introduction}
%
%How local communities are assembled from species pools is an enduring and controversial question in ecology.
%
%The role of determinism has long been recognised: environmental characteristics create niches to which species are differentially adapted, so their probability of successfully colonising any one site depends on ecological traits.
%
%In addition, biotic interactions such as competition (which may be indirect), predation and parasitism limit species' abilities to colonise certain areas. 
%
%While environmental determinism predicts that species in a given locale will be similar, competition between organisms is thought to be more intense with greater similarity, creating a push in the opposite direction. The relative importance of environmental filtering vs biotic interactions varies over spatiotemporal scales and depends on the types of environments considered, and the differences between them.
%
%Isolating the influence of environmental determinism can be difficult. Experiments involving introducing different plant mixes to plots and comparing outcomes.
%
%Migratory birds offer an opportunity. Distinct species pools occur in summer and winter, sharing some but not all species; further, migrants may be dominant members of their (temporary) community. On the other hand, environmental changes are thought to be the impetus for the evolution of migration, as one habitat or the other becomes unsuitable, or the species is unable to compete (Newton).
%
%Therefore, a site near the tropics is optimal, as it is lightly seasonal coinciding with polar seasons, rather than dominated by rainfall. In the southern hemisphere, migrants may be present in both seasons, with boreal breeders in the austal summers replaced by austral breeders wintering at lower latitudes.
%
%As the seasonal alteration occurs within years, community alteration happens rapidly and is not colinear with long-term trends, such as climatic shifts, which trouble the interpretation of long-term vegetation studies.
%
%Birds are also able to self-assemble, and are among the most dispersive organisms on the planet, so studying bird communities is less artificial than planted plots.
%
%Here, we compared seasonal changes in species and functional composition to explore the role of environmental determinism in bird community assembly. While we know that the species pool changes circannually, under environmental determinism we expect functional composition of each community to remain constant.

\newpage
\section{Methods}

\subsection{Survey data}

This study used survey data collected for the CNAA. These censuses were conducted at wetland sites on the coast of Rio Grande do Sul, Brazil. Censuses were conducted during the austral summer (February or March) and winter (July or August), from winter 2005 to summer 2015. 

Some sites were sampled very frequently (almost every season); others less frequently and several only once. Surveys were conducted at varying times of day and for varying durations. The effect of survey effort (duration, number of observers) or other characteristics (such as weather, human disturbance) was neither analysed nor corrected for as these data were not available for all surveys. The analysis therefore does not assume equality of survey effort. 

Surveys which detected fewer than 5 species were excluded from the analysis, as such atypically low numbers are unlikely to be representative of the total community habitually present at a site. 

\subsection{Species data}

Species names and taxonomy followed \cite{vanPerlo2009}. Only species seen on at least 5 surveys were included, as species detected on so few occasions ($<5\%$) are unlikely to be regular members of a given community. 

Species were classified as resident, summer migrants or winter migrants based on the account in \cite{vanPerlo2009}. However, this source contains numerous errors for this region; for example, Semipalmated Plover \textit{Charadrius semipalmatus} `rare or vagrant' species, when in fact it is regularly recorded on the coast in large numbers (e.g. \cite{Sanabria2011}, \cite{Scherer2012} and pers. obs.). Therefore, information on the local status of birds was supplemented using a checklist published for Parque Nacional da Lagoa do Peixe, a large protected site in the study area (\cite{Nascimento1995}).

\subsection{Trait data}	

Variables were selected to cover as many traits related to foraging as possible. Like \cite{Mendez2012} we used a combination of morphological, behavioural and dietary variables.

Data were gathered initially from \cite{DelHoyo2016}; however, this source does not include bill or tarsus lengths, and does not have detailed diet and/or behaviour information for all species (resident species are especially poorly covered). Therefore, where the species account in \cite{DelHoyo2016} reported that information was scarced, additional data were used from other published accounts. In addition, some species have had their diet studied locally at one of the sites surveyed, in which case this information was used to supplement the \cite{DelHoyo2016} account. A full reference list for each trait is available in Appendix 1A.


\subsubsection{\textit{Morphological data}}

\begin{description}
\item[Body mass, Bill length, Tarsus length] Where mean values were reported, these were used; for sexually dimorphic species, the mean of each sex's mean was used to calculate an overall value for the species. Where a range was reported, the median value of this range, or of both ranges for sexually dimorphic species, was used. Data for adult birds was used in preference to juveniles, and local subspecies if geographical variation was reported. These continuous data were $\ln(x)$-transformed prior to analysis as they ranged over an order of magnitude.
\item[Bill shape] As in \cite{Mendez2012}, bills were classified as straight, up-curved or down-curved, based on illustrations in \cite{DelHoyo2016}.
\end{description}
\medskip
\subsubsection{\textit{Behavioural data}} Whether or not the species is recorded as using the following behaviours to forage:

\begin{description}
\item[Pecking] Taking an item from the surface of the substrate
\item[Probing] Inserting the bill below the surface of the sediment
\item[Jabbing] Repeatedly pecking at a prey item
\item[Hammering] Striking repeatedly with the bill in order to break an object
\item[Scything] Moving the bill from side to side through the water or sediment
\item[Turning] Turning over objects to locate prey
\item[Foot trembling] Inserting a foot into the sediment and vibrating it to disturb invertebrates
\item[Swimming] Foraging while swimming on the surface of water
\item[Dipping] Inserting the entire head beneath the surface of the water
\item[Diving] Foraging entirely below the surface of the water
\item[Skimming] Flying with the lower mandible of the bill inserted into the water or sediment
\item[Aerial] Jumping or flying to catch prey before landing
\item[Kleptoparasitism] Stealing prey caught by individuals of other species
\item[Scavenging] Consuming animal matter or waste discarded by other species, including humans
\end{description}
\medskip
\subsubsection{\textit{Diet data}} If the species is recorded as consuming:

\begin{description}
\item[Insects] Adult or larval insects
\item[Crabs, Crustaceans] Crabs (and other large Decapoda, such as shrimp) were considered separately to smaller crustaceans such as Amphipoda
\item[Worms] A highly polyphyletic grouping: any long, legless, soft-bodied invertebrate
\item[Snails, Molluscs] Snails (and other small Gastropoda) were considered separately to larger, harder shelled molluscs such as clams
\item[Herpetofauna] Amphibians and Reptiles
\item[Fish] Fish
\item[Plant] Any plant matter, including leaves, seeds and other parts
\item[Eggs] Eggs of any taxon, including birds, fish and horseshoe crabs
\item[Diatoms] Microscopic, unicelluar phytoplankton characterised by a cell wall composed of silicon dioxide
\item[Other] Any other diet component not included in the above categories, but insufficiently distinct or common to warrant its own; for example, bird nestlings, human rubbish, faeces
\end{description}

If a given behaviour or dietary component was not recorded in the literature, it was assumed not to occur commonly enough to be considered a trait of the species. For one species, \textit{Gallinula melanops}, no diet information is published. For several, no morphological data was available in the literature; however, the analytical methods used are able to function with small amounts of missing data.

\subsection{Distance matrices}

	\textit{Trait differences between species}\\
\\
Due to the inclusion of categorical variables, Gower distance was used for the trait data (\cite{Borcard2011}), using the \texttt{gowdis} function of the \texttt{FD} package for \texttt{R} (\cite{Laliberte2014}). 

In order to give equal weight to the three categories of trait variables (behaviour, diet and morphology), despite their unequal number of variables, I first constructed separate distance matrices for each category. The mean of those matrices gave an overall trait distance matrix. Correlation between these four matrices was tested using Pearson's correlation coefficient (significance tests were not conducted due to non-independence of the overall matrix).

The functional diversity metrics used in this analysis were based on PCoA (\cite{Laliberte2010}) of the trait distance matrix. For these analyses, a Euclidian trait distance matrix is required, calculated from untransformed species abundances. Negative eigenvalues (imaginary axes) were made Euclidian using the Cailliez transformation, which adds the smallest possible constant to the distances (\cite{Laliberte2010}).\\
\\
\textit{Species differences between sites}\\
\\
The functional diversity analyses include species abundances only as a simple weighting. For all other analyses, species dissimilarity between sites used the Cao index implemented by the \texttt{vegdist} function of \texttt{vegan} (\cite{Oksanen2016}). This measure minimises bias in data with high beta diversity and variation in sampling intensity, and therefore has higher success in correctly classifying sites (\cite{Cao1997}). This implementation of the index is based on $\ln(x)$-transformation of abundances; zero counts are arbitrarily replaced with 0.1.

\subsection{Functional diversity metrics}

Functional diversity is a complex and evolving concept. Modern approaches acknowledge that diversity is a multifaceted issue, and that therefore using several metrics in a single analysis is the preferred approach (\cite{Laliberte2010}).

The approach used in this analysis is based on the distance between communities in multivariate trait space (based on the PCoA axes of a Gower distance matrix). The approach used by \cite{Mendez2012}, who used presence/absence data, was based on distances along branches of hierarchical classifications; the approach used here additionally incorporates information about species' relative abundances (\cite{Villeger2008}); I considered this a useful feature for analysing data with abundances ranging from 0 to 4408.

Five independent functional diversity metrics were used in this analysis:

\begin{description}
\item[Functional richness] is the minimum convex-hull volume which can include all species in multivariate space. Proposed as a multivariate analogue of the range of a single trait by \cite{Villeger2008}, it is unweighted by abundance and therefore sensitive to outliers \cite{Laliberte2010}.
\item[Functional evenness] is a description of the abundance distribution within a communities convex hull. Higher values indicate that distances between all nearest-neighbour pairs are similar; values tend towards zero with increasingly clustered points.
\item[Functional divergence] describes the abundance distribution of species relative to the centre of the convex hull. As noted above, the vertices of the convex hull may be formed by common species or by very rare outliers. High values indicate that the abundant species are close to the vertices of the hull; values approach zero as abundant species approach the centre of the hull.
\item[Functional dispersion] was proposed by \cite{Laliberte2010} to combine the strengths of the first three measures. Like evenness and divergence, it incorporates abundance information; unlike those measures, but like richness, it considers the dispersion of species in trait space (rather than within a convex hull \textit{independently} of its volume). In essence, dispersion is the mean distance of each individual species to the centroid of all species in the community.
\item[Number of functionally distinct species] in this analysis is equal to the number of species recorded in a community, as all included species were functionally distinct.
\end{description}

\subsection{Observed functional diversity}

As all the diversity metrics were bounded between 0 and 1 (except for species count data), I used non-parametric testing to compare measures between categories. 

To compare functional diversity between habitats I used Kruskal-Wallis tests to look for significant variation amongst groups, and pairwise Mann-Whitney $U$ tests to identify the source(s) of that variation. As there were only two seasons, these were directly compared with Mann-Whitney $U$ tests.

Due to the multiple comparisons in each analysis, $p$-values were adjusted using the false discovery rate (FDR) correction of \cite{Benjamini1995}.

In addition to these analyses, I tested for an interaction between habitat and seasonal effects using analysis of variance. For each functional diversity metric I tested models allowing different intercepts for each group, and models allowing different slopes and intercepts for each group. Models which included a significant term for at least one habitat and a seasonal effect were considered significant.

\subsection{Null communities and SES}

I also compared the observed functional diversity metrics to the `expected' functional diversity of randomly generated communities.

I first identified the separate species pools for summer and winter surveys. Then, for each survey included in the above analyses I drew the same number of individuals as was observed in the original survey, randomly, from the appropriate seasonal pool. These draws were weighted by the relative frequency of occurrence of each species within that season. This process was repeated to generate 1000 random communities for each survey site.

Following the same methods and trait data used to analyse the observed data (detailed above), all five functional diversity metrics were then calculated for each random community.

As in \cite{Mendez2012}, these values were used to calculate the standardised effect size (SES). \cite{Gotelli2002} give the formula as
\begin{equation}
\frac{I_{obs} - I_{sim}}{\sigma_{sim}},
\end{equation}
where $I_{obs}$ is the observed functional diversity index, $I_{sim}$ is the mean of the 1000 indices calculated from random communities, and $\sigma_{sim}$ is the standard deviation of those 1000 random indices.

If SES does not differ significantly from 0 then observed communities do not differ from random. Positive values indicate trait divergence, for example as might result from interspecific competition; negative values indicate trait convergence, for example as a result of ecological filtering by local habitat characteristics.

\documentclass[12pt,a4paper]{article}\usepackage[]{graphicx}\usepackage[]{color}
%% maxwidth is the original width if it is less than linewidth
%% otherwise use linewidth (to make sure the graphics do not exceed the margin)
\makeatletter
\def\maxwidth{ %
  \ifdim\Gin@nat@width>\linewidth
    \linewidth
  \else
    \Gin@nat@width
  \fi
}
\makeatother

\definecolor{fgcolor}{rgb}{0.345, 0.345, 0.345}
\newcommand{\hlnum}[1]{\textcolor[rgb]{0.686,0.059,0.569}{#1}}%
\newcommand{\hlstr}[1]{\textcolor[rgb]{0.192,0.494,0.8}{#1}}%
\newcommand{\hlcom}[1]{\textcolor[rgb]{0.678,0.584,0.686}{\textit{#1}}}%
\newcommand{\hlopt}[1]{\textcolor[rgb]{0,0,0}{#1}}%
\newcommand{\hlstd}[1]{\textcolor[rgb]{0.345,0.345,0.345}{#1}}%
\newcommand{\hlkwa}[1]{\textcolor[rgb]{0.161,0.373,0.58}{\textbf{#1}}}%
\newcommand{\hlkwb}[1]{\textcolor[rgb]{0.69,0.353,0.396}{#1}}%
\newcommand{\hlkwc}[1]{\textcolor[rgb]{0.333,0.667,0.333}{#1}}%
\newcommand{\hlkwd}[1]{\textcolor[rgb]{0.737,0.353,0.396}{\textbf{#1}}}%

\usepackage{framed}
\makeatletter
\newenvironment{kframe}{%
 \def\at@end@of@kframe{}%
 \ifinner\ifhmode%
  \def\at@end@of@kframe{\end{minipage}}%
  \begin{minipage}{\columnwidth}%
 \fi\fi%
 \def\FrameCommand##1{\hskip\@totalleftmargin \hskip-\fboxsep
 \colorbox{shadecolor}{##1}\hskip-\fboxsep
     % There is no \\@totalrightmargin, so:
     \hskip-\linewidth \hskip-\@totalleftmargin \hskip\columnwidth}%
 \MakeFramed {\advance\hsize-\width
   \@totalleftmargin\z@ \linewidth\hsize
   \@setminipage}}%
 {\par\unskip\endMakeFramed%
 \at@end@of@kframe}
\makeatother

\definecolor{shadecolor}{rgb}{.97, .97, .97}
\definecolor{messagecolor}{rgb}{0, 0, 0}
\definecolor{warningcolor}{rgb}{1, 0, 1}
\definecolor{errorcolor}{rgb}{1, 0, 0}
\newenvironment{knitrout}{}{} % an empty environment to be redefined in TeX

\usepackage{alltt}
\usepackage[utf8]{inputenc}
\usepackage{fullpage} % for margins
% \usepackage{setspace} % for line spacing
% \onehalfspacing % one and a half lines' spacing
\usepackage[cmintegrals,cmbraces]{newtxmath} % maths in that font
\usepackage{ebgaramond-maths} % nice font
\usepackage{amsmath}
\usepackage[T1]{fontenc} % for the Garamond font + math mode
\usepackage{natbib} % extra citation styles
\usepackage{hyperref}
\author{Josh Nightingale}
\IfFileExists{upquote.sty}{\usepackage{upquote}}{}
\begin{document}





\tableofcontents
\clearpage

\section*{Results}



% Table created by stargazer v.5.2 by Marek Hlavac, Harvard University. E-mail: hlavac at fas.harvard.edu
% Date and time: Tue, Jun 28, 2016 - 01:23:20
\begin{table}[tb] \centering 
  \caption{Number of included surveys in each habitat and season category} 
  \label{Ntab} 
\small 
\begin{tabular}{@{\extracolsep{5pt}} cccc} 
\\[-1.8ex]\hline 
\hline \\[-1.8ex] 
 & Summer & Winter & Total \\ 
\hline \\[-1.8ex] 
Beach & $17$ & $10$ & $27$ \\ 
Lake & $48$ & $39$ & $87$ \\ 
Total & $65$ & $49$ & $114$ \\ 
\hline \\[-1.8ex] 
\end{tabular} 
\end{table} 


Out of a total of $141$ surveys, $27$ were excluded due to insufficient observations. This left $114$ surveys and $52$ species that were included in the analysis (Table \ref{Ntab}). 

While over half of surveys were conducted in lake habitats, an adequate ($>20$) number of beach surveys remained in the analysis. The seasonal division between summer and winter was more even, with $57$\% conducted in summer.

\clearpage
\subsection{Distance matrices}

\begin{knitrout}
\definecolor{shadecolor}{rgb}{0.969, 0.969, 0.969}\color{fgcolor}\begin{figure}[t]

{\centering \includegraphics[width=0.7\textwidth,height=0.7\textwidth]{figure/distpairs-1} 

}

\caption[Correlation matrix between the behaviour, diet, morphlogy and mean distance matrices]{Correlation matrix between the behaviour, diet, morphlogy and mean distance matrices. Lower panels show scatter plots; upper panels show Pearson correlation coefficients, scaled according to their value.}\label{fig:distpairs}
\end{figure}


\end{knitrout}


The three distance sub-matrices (behaviour, diet and morphology) were poorly correlated with each other ($r \approx 0.1$), suggesting that these variables captured different aspects of species' functional ecology. 
However, the overall distance matrix used in subsequent analysis, calculated from the mean of the three sub-matrices, had a Pearson correlation $r > 0.5$ with all three sub-matrices, suggesting that all of these aspects were represented in the matrix used for analysis (Figure \ref{fig:distpairs}).

\clearpage
\subsection{Predicting community composition and function}



Species assemblages were highly significantly more dissimilar between sites in different habitats ($p < 0.001$), with this relationship explaining $38.1$\% of variation. In addition, season was a significant predictor of community dissimilarity, though with very weak expanatory power ($R^2 = 0.001$). The distance matrix including both habitat and season was also a very significant predictor of community dissimilarity ($p < 0.001$), with intermediate $F$-statistic and $R^2$ (Table \ref{comp_pred_sg}). Differences in each species' occurrence and abundance between habitats are shown by Table \ref{hab_com_tab} at the end of the results section.


% Table created by stargazer v.5.2 by Marek Hlavac, Harvard University. E-mail: hlavac at fas.harvard.edu
% Date and time: Tue, Jun 28, 2016 - 01:23:22
% Requires LaTeX packages: dcolumn 
\begin{table}[tb] \centering 
  \caption{Predicting inter-survey dissimilarity of species composition with three regression models using distance matrices of (1) Survey habitat, (2) Season of survey and (3) Habitat and season. Parameter estimates are presented with their 95\% confidence intervals} 
  \label{comp_pred_sg} 
\small 
\begin{tabular}{@{\extracolsep{5pt}}lD{.}{.}{-2} D{.}{.}{-2} D{.}{.}{-2} } 
\\[-1.8ex]\hline 
\hline \\[-1.8ex] 
 & \multicolumn{3}{c}{\textit{Dependent variable:}} \\ 
\cline{2-4} 
\\[-1.8ex] & \multicolumn{3}{c}{Species composition} \\ 
\\[-1.8ex] & \multicolumn{1}{c}{(1)} & \multicolumn{1}{c}{(2)} & \multicolumn{1}{c}{(3)}\\ 
\hline \\[-1.8ex] 
 Habitat & 0.28^{***} &  &  \\ 
  & \multicolumn{1}{c}{(0.28$, $0.29)} &  &  \\ 
  & & & \\ 
 Season &  & 0.01^{**} &  \\ 
  &  & \multicolumn{1}{c}{(0.004$, $0.03)} &  \\ 
  & & & \\ 
 Habitat and season &  &  & 0.29^{***} \\ 
  &  &  & \multicolumn{1}{c}{(0.28$, $0.31)} \\ 
  & & & \\ 
 Intercept & 1.06^{***} & 1.15^{***} & 1.03^{***} \\ 
  & \multicolumn{1}{c}{(1.05$, $1.06)} & \multicolumn{1}{c}{(1.15$, $1.16)} & \multicolumn{1}{c}{(1.03$, $1.04)} \\ 
  & & & \\ 
\hline \\[-1.8ex] 
Observations & \multicolumn{1}{c}{6,441} & \multicolumn{1}{c}{6,441} & \multicolumn{1}{c}{6,441} \\ 
R$^{2}$ & \multicolumn{1}{c}{0.38} & \multicolumn{1}{c}{0.001} & \multicolumn{1}{c}{0.21} \\ 
F Statistic (df = 1; 6439) & \multicolumn{1}{c}{3,965.34$^{***}$} & \multicolumn{1}{c}{6.85$^{**}$} & \multicolumn{1}{c}{1,675.27$^{***}$} \\ 
\hline 
\hline \\[-1.8ex] 
\textit{Note:}  & \multicolumn{3}{r}{$^{*}$p$<$0.05; $^{**}$p$<$0.01; $^{***}$p$<$0.001} \\ 
\end{tabular} 
\end{table} 


Differences in community functional composition were also strongly related to habitat differences ($p < 0.001$), though with much less variation explained by this relationship ($R^2 = 0.199$; Table \ref{func_pred_sg}). 

Unlike species composition, functional composition did not vary between seasons. Attempts to model this relationship resulted in an $F$-statistic and $R^2$ very close to $0$. The matrix including habitat and season was a significant predictor of species dissimilarity ($p < 0.001$), but with lower $F$-statistic and $R^2$ than the habitat-only model (Table \ref{func_pred_sg}).

Considering each category of functional traits separately confirmed that birds' behaviour, diet and morphology all vary significantly between habitats, but none between seasons (Table \ref{func_subs_pred_sg}). Indeed, even individual traits analysed in isolation often varied significantly ($p < 0.05$) between habitats. Relative frequencies of foraging behaviours and diet items are presented and analysed in Table \ref{habcwmtab}.


% Table created by stargazer v.5.2 by Marek Hlavac, Harvard University. E-mail: hlavac at fas.harvard.edu
% Date and time: Tue, Jun 28, 2016 - 01:23:22
% Requires LaTeX packages: dcolumn 
\begin{table}[!htbp] \centering 
  \caption{Predicting inter-survey dissimilarity of functional composition (Community Weighted Means) with three regression models using distance matrices of (1) Survey habitat, (2) Season of survey and (3) Habitat and season. Parameter estimates are presented with their 95\% confidence intervals} 
  \label{func_pred_sg} 
\small 
\begin{tabular}{@{\extracolsep{5pt}}lD{.}{.}{-2} D{.}{.}{-2} D{.}{.}{-2} } 
\\[-1.8ex]\hline 
\hline \\[-1.8ex] 
 & \multicolumn{3}{c}{\textit{Dependent variable:}} \\ 
\cline{2-4} 
\\[-1.8ex] & \multicolumn{3}{c}{Functional composition} \\ 
\\[-1.8ex] & \multicolumn{1}{c}{(1)} & \multicolumn{1}{c}{(2)} & \multicolumn{1}{c}{(3)}\\ 
\hline \\[-1.8ex] 
 Habitat & 0.08^{***} &  &  \\ 
  & \multicolumn{1}{c}{(0.08$, $0.09)} &  &  \\ 
  & & & \\ 
 Season &  & -0.001 &  \\ 
  &  & \multicolumn{1}{c}{(-0.005$, $0.004)} &  \\ 
  & & & \\ 
 Habitat and season &  &  & 0.08^{***} \\ 
  &  &  & \multicolumn{1}{c}{(0.07$, $0.09)} \\ 
  & & & \\ 
 Intercept & 0.21^{***} & 0.24^{***} & 0.20^{***} \\ 
  & \multicolumn{1}{c}{(0.20$, $0.21)} & \multicolumn{1}{c}{(0.23$, $0.24)} & \multicolumn{1}{c}{(0.20$, $0.21)} \\ 
  & & & \\ 
\hline \\[-1.8ex] 
Observations & \multicolumn{1}{c}{6,441} & \multicolumn{1}{c}{6,441} & \multicolumn{1}{c}{6,441} \\ 
R$^{2}$ & \multicolumn{1}{c}{0.20} & \multicolumn{1}{c}{0.0000} & \multicolumn{1}{c}{0.10} \\ 
F Statistic (df = 1; 6439) & \multicolumn{1}{c}{1,598.57$^{***}$} & \multicolumn{1}{c}{0.07} & \multicolumn{1}{c}{678.62$^{***}$} \\ 
\hline 
\hline \\[-1.8ex] 
\textit{Note:}  & \multicolumn{3}{r}{$^{*}$p$<$0.05; $^{**}$p$<$0.01; $^{***}$p$<$0.001} \\ 
\end{tabular} 
\end{table} 



% Table created by stargazer v.5.2 by Marek Hlavac, Harvard University. E-mail: hlavac at fas.harvard.edu
% Date and time: Tue, Jun 28, 2016 - 01:23:23
% Requires LaTeX packages: dcolumn 
\begin{table}[!htbp] \centering 
  \caption{Predicting sub-categories of community functional composition using linear models with survey habitat (odd numbers) or season (even numbers) as predictor variables} 
  \label{func_subs_pred_sg} 
\small 
\begin{tabular}{@{\extracolsep{5pt}}lD{.}{.}{-1} D{.}{.}{-1} D{.}{.}{-1} D{.}{.}{-1} D{.}{.}{-1} D{.}{.}{-1} } 
\\[-1.8ex]\hline 
\hline \\[-1.8ex] 
 & \multicolumn{6}{c}{\textit{Dependent variable:}} \\ 
\cline{2-7} 
\\[-1.8ex] & \multicolumn{2}{c}{Behaviour} & \multicolumn{2}{c}{Diet} & \multicolumn{2}{c}{Morphology} \\ 
\\[-1.8ex] & \multicolumn{1}{c}{(1)} & \multicolumn{1}{c}{(2)} & \multicolumn{1}{c}{(3)} & \multicolumn{1}{c}{(4)} & \multicolumn{1}{c}{(5)} & \multicolumn{1}{c}{(6)}\\ 
\hline \\[-1.8ex] 
 Habitat & 0.1^{***} &  & 0.1^{***} &  & 0.1^{***} &  \\ 
  Season &  & 0.002 &  & -0.002 &  & -0.001 \\ 
  Intercept & 0.2^{***} & 0.2^{***} & 0.2^{***} & 0.3^{***} & 0.2^{***} & 0.2^{***} \\ 
 \hline \\[-1.8ex] 
Observations & \multicolumn{1}{c}{6,441} & \multicolumn{1}{c}{6,441} & \multicolumn{1}{c}{6,441} & \multicolumn{1}{c}{6,441} & \multicolumn{1}{c}{6,441} & \multicolumn{1}{c}{6,441} \\ 
R$^{2}$ & \multicolumn{1}{c}{0.2} & \multicolumn{1}{c}{0.000} & \multicolumn{1}{c}{0.3} & \multicolumn{1}{c}{0.000} & \multicolumn{1}{c}{0.02} & \multicolumn{1}{c}{0.000} \\ 
F Statistic & \multicolumn{1}{c}{1,706.2$^{***}$} & \multicolumn{1}{c}{0.4} & \multicolumn{1}{c}{2,182.6$^{***}$} & \multicolumn{1}{c}{0.8} & \multicolumn{1}{c}{146.9$^{***}$} & \multicolumn{1}{c}{0.1} \\ 
\hline 
\hline \\[-1.8ex] 
\textit{Note:}  & \multicolumn{6}{r}{$^{*}$p$<$0.05; $^{**}$p$<$0.01; $^{***}$p$<$0.001} \\ 
\end{tabular} 
\end{table} 





% Table created by stargazer v.5.2 by Marek Hlavac, Harvard University. E-mail: hlavac at fas.harvard.edu
% Date and time: Tue, Jun 28, 2016 - 01:23:24
% Requires LaTeX packages: dcolumn 
\begin{table}[tb] \centering 
  \caption{Inter-habitat variation in dominant species traits (Community Weighted Means). For diet and behaviour traits, the value shown is the percentage of individuals in the community possessing that trait. The equality of proportions of species having a given trait in each habitat was tested with a series of $\chi^2$ tests (df$=1$), the results of which are presented here with FDR-adjusted $p$-values. Data are presented in order of increasing statistical significance. Mass was measured in grams (g), tarsus and bill lengths in millimetres (mm); though these values were $\ln(x)$ transformed for analysis, untransformed data are presented here. } 
  \label{habcwmtab} 
\small 
\begin{tabular}{@{\extracolsep{5pt}} D{.}{.}{-3} D{.}{.}{-3} D{.}{.}{-3} D{.}{.}{-3} D{.}{.}{-3} } 
\\[-1.8ex]\hline 
\hline \\[-1.8ex] 
\multicolumn{1}{c}{Trait} & \multicolumn{1}{c}{Beach} & \multicolumn{1}{c}{Lake} & \multicolumn{1}{c}{Chi.sq} & \multicolumn{1}{c}{p.value} \\ 
\hline \\[-1.8ex] 
\multicolumn{1}{c}{Plant} & \multicolumn{1}{c}{15.4} & \multicolumn{1}{c}{77.4} & \multicolumn{1}{c}{74.8} & \multicolumn{1}{c}{\textless 0.001} \\ 
\multicolumn{1}{c}{Crabs} & \multicolumn{1}{c}{53.9} & \multicolumn{1}{c}{5.4} & \multicolumn{1}{c}{54.1} & \multicolumn{1}{c}{\textless 0.001} \\ 
\multicolumn{1}{c}{Dipping} & \multicolumn{1}{c}{5.8} & \multicolumn{1}{c}{54.8} & \multicolumn{1}{c}{54.5} & \multicolumn{1}{c}{\textless 0.001} \\ 
\multicolumn{1}{c}{Snails} & \multicolumn{1}{c}{55.2} & \multicolumn{1}{c}{6.7} & \multicolumn{1}{c}{52.8} & \multicolumn{1}{c}{\textless 0.001} \\ 
\multicolumn{1}{c}{Scavenging} & \multicolumn{1}{c}{45.9} & \multicolumn{1}{c}{2.4} & \multicolumn{1}{c}{49.3} & \multicolumn{1}{c}{\textless 0.001} \\ 
\multicolumn{1}{c}{Scything} & \multicolumn{1}{c}{6.5} & \multicolumn{1}{c}{49.7} & \multicolumn{1}{c}{44.1} & \multicolumn{1}{c}{\textless 0.001} \\ 
\multicolumn{1}{c}{Other} & \multicolumn{1}{c}{42.9} & \multicolumn{1}{c}{5.8} & \multicolumn{1}{c}{35.4} & \multicolumn{1}{c}{\textless 0.001} \\ 
\multicolumn{1}{c}{Eggs} & \multicolumn{1}{c}{41.3} & \multicolumn{1}{c}{8.4} & \multicolumn{1}{c}{27.2} & \multicolumn{1}{c}{\textless 0.001} \\ 
\multicolumn{1}{c}{Kleptoparasitism} & \multicolumn{1}{c}{29.1} & \multicolumn{1}{c}{4.3} & \multicolumn{1}{c}{20.4} & \multicolumn{1}{c}{\textless 0.001} \\ 
\multicolumn{1}{c}{Worms} & \multicolumn{1}{c}{69} & \multicolumn{1}{c}{36.3} & \multicolumn{1}{c}{20.2} & \multicolumn{1}{c}{\textless 0.001} \\ 
\multicolumn{1}{c}{Swimming} & \multicolumn{1}{c}{23.2} & \multicolumn{1}{c}{52} & \multicolumn{1}{c}{16.5} & \multicolumn{1}{c}{\textless 0.001} \\ 
\multicolumn{1}{c}{Herps} & \multicolumn{1}{c}{17.4} & \multicolumn{1}{c}{41.7} & \multicolumn{1}{c}{13} & \multicolumn{1}{c}{0.001} \\ 
\multicolumn{1}{c}{Molluscs} & \multicolumn{1}{c}{66.3} & \multicolumn{1}{c}{41} & \multicolumn{1}{c}{11.9} & \multicolumn{1}{c}{0.001} \\ 
\multicolumn{1}{c}{Foot trembling} & \multicolumn{1}{c}{25.3} & \multicolumn{1}{c}{6.5} & \multicolumn{1}{c}{11.8} & \multicolumn{1}{c}{0.001} \\ 
\multicolumn{1}{c}{Hammering} & \multicolumn{1}{c}{11.5} & \multicolumn{1}{c}{0.1} & \multicolumn{1}{c}{9.9} & \multicolumn{1}{c}{0.003} \\ 
\multicolumn{1}{c}{Diving} & \multicolumn{1}{c}{39.6} & \multicolumn{1}{c}{62.4} & \multicolumn{1}{c}{9.5} & \multicolumn{1}{c}{0.003} \\ 
\multicolumn{1}{c}{Insects} & \multicolumn{1}{c}{96.5} & \multicolumn{1}{c}{82.2} & \multicolumn{1}{c}{9.3} & \multicolumn{1}{c}{0.003} \\ 
\multicolumn{1}{c}{Jabbing} & \multicolumn{1}{c}{12.1} & \multicolumn{1}{c}{2.1} & \multicolumn{1}{c}{6.1} & \multicolumn{1}{c}{0.018} \\ 
\multicolumn{1}{c}{Probing} & \multicolumn{1}{c}{40.9} & \multicolumn{1}{c}{24.6} & \multicolumn{1}{c}{5.3} & \multicolumn{1}{c}{0.028} \\ 
\multicolumn{1}{c}{Fish} & \multicolumn{1}{c}{67.8} & \multicolumn{1}{c}{51.1} & \multicolumn{1}{c}{5.1} & \multicolumn{1}{c}{0.03} \\ 
\multicolumn{1}{c}{Aerial} & \multicolumn{1}{c}{12.2} & \multicolumn{1}{c}{3.2} & \multicolumn{1}{c}{4.5} & \multicolumn{1}{c}{0.04} \\ 
\multicolumn{1}{c}{Crustaceans} & \multicolumn{1}{c}{82.2} & \multicolumn{1}{c}{68.5} & \multicolumn{1}{c}{4.3} & \multicolumn{1}{c}{0.04} \\ 
\multicolumn{1}{c}{Pecking} & \multicolumn{1}{c}{49.4} & \multicolumn{1}{c}{65} & \multicolumn{1}{c}{4.4} & \multicolumn{1}{c}{0.04} \\ 
\multicolumn{1}{c}{Turning} & \multicolumn{1}{c}{0} & \multicolumn{1}{c}{0.8} & \multicolumn{1}{c}{0} & \multicolumn{1}{c}{1} \\ 
\multicolumn{1}{c}{Skimming} & \multicolumn{1}{c}{2.9} & \multicolumn{1}{c}{2.6} & \multicolumn{1}{c}{0} & \multicolumn{1}{c}{1} \\ 
\multicolumn{1}{c}{Mass} & \multicolumn{1}{c}{222.3} & \multicolumn{1}{c}{625.4} & \multicolumn{1}{c}{-} & \multicolumn{1}{c}{-} \\ 
\multicolumn{1}{c}{Bill length} & \multicolumn{1}{c}{39.9} & \multicolumn{1}{c}{58} & \multicolumn{1}{c}{-} & \multicolumn{1}{c}{-} \\ 
\multicolumn{1}{c}{Tarsus} & \multicolumn{1}{c}{40.5} & \multicolumn{1}{c}{60.7} & \multicolumn{1}{c}{-} & \multicolumn{1}{c}{-} \\ 
\multicolumn{1}{c}{Bill shape} & \multicolumn{1}{c}{Straight} & \multicolumn{1}{c}{Straight} & \multicolumn{1}{c}{-} & \multicolumn{1}{c}{-} \\ 
\multicolumn{1}{c}{Diatoms} & \multicolumn{1}{c}{0} & \multicolumn{1}{c}{0} & \multicolumn{1}{c}{-} & \multicolumn{1}{c}{-} \\ 
\hline \\[-1.8ex] 
\end{tabular} 
\end{table} 


\clearpage
\subsection{Austral \emph{versus} boreal migrant communities} % change title

In all surveys analysed, $23.1$\% of the $52$ species recorded were migratory, comprising $30.9$\% of all individuals recorded. Of the migratory species, $3$ were austral migrants (present in the study area during the austral winter) and $9$ were boreal migrants (locally present during summer).

In total, $27$ surveys were included in the analyses comparing functional traits amongst migrants, and between migrant and resident species. The distribution of these surveys amongst habitats and seasons is shown by Table \ref{comcompNtab}.


% Table created by stargazer v.5.2 by Marek Hlavac, Harvard University. E-mail: hlavac at fas.harvard.edu
% Date and time: Tue, Jun 28, 2016 - 01:23:25
% Requires LaTeX packages: dcolumn 
\begin{table}[tb] \centering 
  \caption{Sample sizes by survey habitat and season for bird census included in the comparative community analyses} 
  \label{comcompNtab} 
\small 
\begin{tabular}{@{\extracolsep{5pt}} D{.}{.}{-3} D{.}{.}{-3} D{.}{.}{-3} D{.}{.}{-3} } 
\\[-1.8ex]\hline 
\hline \\[-1.8ex] 
\multicolumn{1}{c}{} & \multicolumn{1}{c}{Summer} & \multicolumn{1}{c}{Winter} & \multicolumn{1}{c}{Total} \\ 
\hline \\[-1.8ex] 
\multicolumn{1}{c}{Beach} & 12 & 7 & 19 \\ 
\multicolumn{1}{c}{Lake} & 7 & 1 & 8 \\ 
\multicolumn{1}{c}{Total} & 19 & 8 & 27 \\ 
\hline \\[-1.8ex] 
\end{tabular} 
\end{table} 


There were significant differences in the fucntional ecology of summer \emph{versus} winter migrants, and between migratory birds using each habitat (Table \ref{migrestraitlm}). Considering the functional subcategories, migratory birds differed significantly in their diet and foraging behaviour between seasons, and in their morphology between habitats (Table \ref{migrestrait_sub}).


% Table created by stargazer v.5.2 by Marek Hlavac, Harvard University. E-mail: hlavac at fas.harvard.edu
% Date and time: Tue, Jun 28, 2016 - 01:23:25
% Requires LaTeX packages: dcolumn 
\begin{table}[tb] \centering 
  \caption{Predicting inter-survey dissimilarity of functional composition of migratory bird communities with three regression models using distance matrices of (1) Survey habitat, (2) Season of survey and (3) Habitat and season. Parameter estimates are presented with their 95\% confidence intervals} 
  \label{migrestraitlm} 
\small 
\begin{tabular}{@{\extracolsep{5pt}}lD{.}{.}{-2} D{.}{.}{-2} D{.}{.}{-2} } 
\\[-1.8ex]\hline 
\hline \\[-1.8ex] 
 & \multicolumn{3}{c}{\textit{Dependent variable:}} \\ 
\cline{2-4} 
\\[-1.8ex] & \multicolumn{3}{c}{Functional composition} \\ 
\\[-1.8ex] & \multicolumn{1}{c}{(1)} & \multicolumn{1}{c}{(2)} & \multicolumn{1}{c}{(3)}\\ 
\hline \\[-1.8ex] 
 Habitat & 0.05^{***} &  &  \\ 
  & \multicolumn{1}{c}{(0.03$, $0.07)} &  &  \\ 
  & & & \\ 
 Season &  & 0.03^{*} &  \\ 
  &  & \multicolumn{1}{c}{(0.01$, $0.05)} &  \\ 
  & & & \\ 
 Habitat and season &  &  & 0.08^{***} \\ 
  &  &  & \multicolumn{1}{c}{(0.05$, $0.11)} \\ 
  & & & \\ 
 Intercept & 0.24^{***} & 0.25^{***} & 0.22^{***} \\ 
  & \multicolumn{1}{c}{(0.22$, $0.25)} & \multicolumn{1}{c}{(0.23$, $0.26)} & \multicolumn{1}{c}{(0.21$, $0.24)} \\ 
  & & & \\ 
\hline \\[-1.8ex] 
Observations & \multicolumn{1}{c}{351} & \multicolumn{1}{c}{351} & \multicolumn{1}{c}{351} \\ 
R$^{2}$ & \multicolumn{1}{c}{0.05} & \multicolumn{1}{c}{0.02} & \multicolumn{1}{c}{0.07} \\ 
F Statistic (df = 1; 349) & \multicolumn{1}{c}{19.14$^{***}$} & \multicolumn{1}{c}{6.65$^{*}$} & \multicolumn{1}{c}{26.45$^{***}$} \\ 
\hline 
\hline \\[-1.8ex] 
\textit{Note:}  & \multicolumn{3}{r}{$^{*}$p$<$0.05; $^{**}$p$<$0.01; $^{***}$p$<$0.001} \\ 
\end{tabular} 
\end{table} 



% Table created by stargazer v.5.2 by Marek Hlavac, Harvard University. E-mail: hlavac at fas.harvard.edu
% Date and time: Tue, Jun 28, 2016 - 01:23:25
% Requires LaTeX packages: dcolumn 
\begin{table}[tb] \centering 
  \caption{Predicting sub-categories of migratory bird communities' functional composition using linear models with survey habitat (odd numbers) or season (even numbers) as predictor variables} 
  \label{migrestrait_sub} 
\small 
\begin{tabular}{@{\extracolsep{5pt}}lD{.}{.}{-1} D{.}{.}{-1} D{.}{.}{-1} D{.}{.}{-1} D{.}{.}{-1} D{.}{.}{-1} } 
\\[-1.8ex]\hline 
\hline \\[-1.8ex] 
 & \multicolumn{6}{c}{\textit{Dependent variable:}} \\ 
\cline{2-7} 
\\[-1.8ex] & \multicolumn{2}{c}{Behaviour} & \multicolumn{2}{c}{Diet} & \multicolumn{2}{c}{Morphology} \\ 
\\[-1.8ex] & \multicolumn{1}{c}{(1)} & \multicolumn{1}{c}{(2)} & \multicolumn{1}{c}{(3)} & \multicolumn{1}{c}{(4)} & \multicolumn{1}{c}{(5)} & \multicolumn{1}{c}{(6)}\\ 
\hline \\[-1.8ex] 
 Habitat & 0.04^{**} &  & -0.001 &  & 0.1^{***} &  \\ 
  Season &  & 0.04^{***} &  & 0.04^{***} &  & 0.004 \\ 
  Intercept & 0.2^{***} & 0.2^{***} & 0.3^{***} & 0.3^{***} & 0.2^{***} & 0.3^{***} \\ 
 \hline \\[-1.8ex] 
Observations & \multicolumn{1}{c}{351} & \multicolumn{1}{c}{351} & \multicolumn{1}{c}{351} & \multicolumn{1}{c}{351} & \multicolumn{1}{c}{351} & \multicolumn{1}{c}{351} \\ 
R$^{2}$ & \multicolumn{1}{c}{0.03} & \multicolumn{1}{c}{0.04} & \multicolumn{1}{c}{0.000} & \multicolumn{1}{c}{0.04} & \multicolumn{1}{c}{0.1} & \multicolumn{1}{c}{0.000} \\ 
F Statistic & \multicolumn{1}{c}{10.4$^{**}$} & \multicolumn{1}{c}{14.0$^{***}$} & \multicolumn{1}{c}{0.02} & \multicolumn{1}{c}{13.5$^{***}$} & \multicolumn{1}{c}{21.0$^{***}$} & \multicolumn{1}{c}{0.03} \\ 
\hline 
\hline \\[-1.8ex] 
\textit{Note:}  & \multicolumn{6}{r}{$^{*}$p$<$0.05; $^{**}$p$<$0.01; $^{***}$p$<$0.001} \\ 
\end{tabular} 
\end{table} 


\clearpage
\subsection{Migrant \emph{versus} resident bird communities: Dominant traits}

Overall, using the same surveys as the above analysis (Table \ref{comcompNtab}), I found that $18$ of $28$ trait variables differed significantly between the migratory and resident species (Table \ref{migrestrait}). Behavioural, diet and morphological traits all included highly significant ($p < 0.001$) differences. While all morphological traits were highly sigificantly different ($p \leq 0.001$), there were diet and behaviour traits that did not differ between migrant and resident birds (Table \ref{migrestrait}). 



Common and rare traits exhibited significant differences. There was no correlation between trait frequency and $p$-value for migrants (Spearman's $r = 0.02$, $p = 0.9$) nor residents (Spearman's $r = \ensuremath{-0.4}$, $p = 0.1$). Note that this analysis included only those traits whose values can be summarised as frequencies (\emph{i.e.}, morphological traits are excluded). 

Instead, traits that were relatively common in migrants tended also to be relatively common in resident species, and \emph{vice-versa} (Spearman's $r = 0.62$, $p = 0.001$; $n=25$ for all tests). However, rare traits in migrants tended to be rarer, and common traits nearer to universal, than in resident species, which showed a more even distribution of trait frequencies (Figure \ref{fig:trait_accum}).

\begin{knitrout}
\definecolor{shadecolor}{rgb}{0.969, 0.969, 0.969}\color{fgcolor}\begin{figure}[bt]

{\centering \includegraphics[width=\textwidth,height=0.6\textwidth]{figure/trait_accum-1} 

}

\caption[Trait accumulation profiles for migrant and resident waterbirds]{Trait accumulation profiles for migrant and resident waterbirds. Plots show the absolute frequency of each behavioural and diet trait, ranked by frequency}\label{fig:trait_accum}
\end{figure}


\end{knitrout}

However, in addition to the data presented in Table \ref{migrestrait}, there was no significant difference in the commonest bill shape, which was `straight' for nearly all communities (data not shown).


% Table created by stargazer v.5.2 by Marek Hlavac, Harvard University. E-mail: hlavac at fas.harvard.edu
% Date and time: Tue, Jun 28, 2016 - 01:23:26
% Requires LaTeX packages: dcolumn 
\begin{table}[tb] \centering 
  \caption{Mean trait values for migratory and resident birds, the difference between those means and the FDR-adjusted $p$-values from Mann-Whitney $U$ tests of the difference between communities. Data are presented in order of increasing statistical significance. For diet and behaviour traits, the value shown is the percentage of individuals in the community possessing that trait. Mass was measured in grams (g), tarsus and bill lengths in millimetres (mm); though these values were $\ln(x)$ transformed for analysis, untransformed data are presented here} 
  \label{migrestrait} 
\small 
\begin{tabular}{@{\extracolsep{5pt}} D{.}{.}{-3} D{.}{.}{-3} D{.}{.}{-3} D{.}{.}{-3} D{.}{.}{-3} } 
\\[-1.8ex]\hline 
\hline \\[-1.8ex] 
\multicolumn{1}{c}{Trait} & \multicolumn{1}{c}{Migrants} & \multicolumn{1}{c}{Residents} & \multicolumn{1}{c}{Difference} & \multicolumn{1}{c}{p.value} \\ 
\hline \\[-1.8ex] 
\multicolumn{1}{c}{Mass} & 129 & 527 & -398 & \multicolumn{1}{c}{\textless 0.001} \\ 
\multicolumn{1}{c}{Bill length} & 34 & 58 & -25 & \multicolumn{1}{c}{\textless 0.001} \\ 
\multicolumn{1}{c}{Tarsus} & 35 & 65 & -30 & \multicolumn{1}{c}{\textless 0.001} \\ 
\multicolumn{1}{c}{Insects} & 99 & 82 & 17 & \multicolumn{1}{c}{\textless 0.001} \\ 
\multicolumn{1}{c}{Crustaceans} & 100 & 64 & 36 & \multicolumn{1}{c}{\textless 0.001} \\ 
\multicolumn{1}{c}{Snails} & 70 & 24 & 46 & \multicolumn{1}{c}{\textless 0.001} \\ 
\multicolumn{1}{c}{Herps} & 0 & 44 & -44 & \multicolumn{1}{c}{\textless 0.001} \\ 
\multicolumn{1}{c}{Hammering} & 0 & 16 & -16 & \multicolumn{1}{c}{\textless 0.001} \\ 
\multicolumn{1}{c}{Swimming} & 6 & 52 & -46 & \multicolumn{1}{c}{\textless 0.001} \\ 
\multicolumn{1}{c}{Skimming} & 0 & 15 & -15 & \multicolumn{1}{c}{\textless 0.001} \\ 
\multicolumn{1}{c}{Pecking} & 39 & 62 & -23 & \multicolumn{1}{c}{0.003} \\ 
\multicolumn{1}{c}{Kleptoparasitism} & 13 & 43 & -30 & \multicolumn{1}{c}{0.003} \\ 
\multicolumn{1}{c}{Foot trembling} & 13 & 42 & -29 & \multicolumn{1}{c}{0.004} \\ 
\multicolumn{1}{c}{Jabbing} & 4 & 18 & -14 & \multicolumn{1}{c}{0.005} \\ 
\multicolumn{1}{c}{Diving} & 24 & 52 & -28 & \multicolumn{1}{c}{0.009} \\ 
\multicolumn{1}{c}{Fish} & 49 & 80 & -30 & \multicolumn{1}{c}{0.01} \\ 
\multicolumn{1}{c}{Other} & 23 & 38 & -15 & \multicolumn{1}{c}{0.033} \\ 
\multicolumn{1}{c}{Probing} & 48 & 26 & 22 & \multicolumn{1}{c}{0.044} \\ 
\multicolumn{1}{c}{Worms} & 56 & 70 & -13 & \multicolumn{1}{c}{0.099} \\ 
\multicolumn{1}{c}{Scything} & 22 & 11 & 10 & \multicolumn{1}{c}{0.183} \\ 
\multicolumn{1}{c}{Aerial} & 12 & 12 & 0 & \multicolumn{1}{c}{0.301} \\ 
\multicolumn{1}{c}{Eggs} & 30 & 25 & 5 & \multicolumn{1}{c}{0.423} \\ 
\multicolumn{1}{c}{Crabs} & 42 & 36 & 6 & \multicolumn{1}{c}{0.466} \\ 
\multicolumn{1}{c}{Molluscs} & 48 & 52 & -5 & \multicolumn{1}{c}{0.466} \\ 
\multicolumn{1}{c}{Plant} & 25 & 19 & 6 & \multicolumn{1}{c}{0.478} \\ 
\multicolumn{1}{c}{Scavenging} & 32 & 31 & 1 & \multicolumn{1}{c}{0.697} \\ 
\multicolumn{1}{c}{Dipping} & 10 & 9 & 1 & \multicolumn{1}{c}{0.838} \\ 
\multicolumn{1}{c}{Turning} & 0 & 0 & 0 & \multicolumn{1}{c}{1} \\ 
\hline \\[-1.8ex] 
\end{tabular} 
\end{table} 









%%%%%%%%%%%%%%%%%%%%%%%%%%%%%%%%%%
%%% functional diversity stuff %%%
%%%%%%%%%%%%%%%%%%%%%%%%%%%%%%%%%%
\clearpage
\subsection{Observed functional diversity}





In general, functional diversity appeared similar between habitats (Figure \ref{fig:fdsea}). Formal hypothesis testing confirmed this similarity. Only functional evenness varied between habitats (Table \ref{fdhabt}). PairwiseMann-Whitney $U$ tests with the Benjamini-Hochberg (1995) correction showed that beach communities were less functionally even than lake ($p = 0$) communities.


% Table created by stargazer v.5.2 by Marek Hlavac, Harvard University. E-mail: hlavac at fas.harvard.edu
% Date and time: Tue, Jun 28, 2016 - 01:23:35
\begin{table}[tb] \centering 
  \caption{FDR-adjusted p values from Mann-Whitney $U$ tests comparing each functional diversity metric between beach and lake survey locations} 
  \label{fdhabt} 
\small 
\begin{tabular}{@{\extracolsep{5pt}} cc} 
\\[-1.8ex]\hline 
\hline \\[-1.8ex] 
Metric & p value \\ 
\hline \\[-1.8ex] 
Dispersion & $0.899$ \\ 
Divergence & $0.451$ \\ 
Evenness & $0.00000$ \\ 
No.Species & $0.451$ \\ 
Richness & $0.00003$ \\ 
\hline \\[-1.8ex] 
\end{tabular} 
\end{table} 


There was no apparent difference in any functional diversity metric between seasons (Figure \ref{fig:fdsea}), which was confirmed by formal hypothesis testing (Table \ref{fdseat}): there was no significant difference in any metric between summer and winter (FDR-adjusted Mann-Whitney $U$ tests: all $p > 0.05$). None of the two-way ANOVA models showed a significant interaction between habitat and season.

\begin{knitrout}
\definecolor{shadecolor}{rgb}{0.969, 0.969, 0.969}\color{fgcolor}\begin{figure}[b]

{\centering \includegraphics[width=\textwidth,height=\textwidth]{figure/fdsea-1} 

}

\caption[Functional diversity metrics according to habitat (left) and season (right)]{Functional diversity metrics according to habitat (left) and season (right). Note that the $y$-axis has a different scale in each plot facet. Only functional evenness by habitat showed a significant difference, with beach communities significantly less even than the others (see text for statistics).}\label{fig:fdsea}
\end{figure}


\end{knitrout}


% Table created by stargazer v.5.2 by Marek Hlavac, Harvard University. E-mail: hlavac at fas.harvard.edu
% Date and time: Tue, Jun 28, 2016 - 01:23:39
\begin{table}[bt] \centering 
  \caption{FDR-adjusted p values from Mann-Whitney tests comparing each functional diversity metric between summer and winter surveys} 
  \label{fdseat} 
\small 
\begin{tabular}{@{\extracolsep{5pt}} cc} 
\\[-1.8ex]\hline 
\hline \\[-1.8ex] 
Metric & p value \\ 
\hline \\[-1.8ex] 
Dispersion & $0.76$ \\ 
Divergence & $0.76$ \\ 
Evenness & $0.76$ \\ 
No.Species & $0.90$ \\ 
Richness & $0.89$ \\ 
\hline \\[-1.8ex] 
\end{tabular} 
\end{table} 






\clearpage
\subsection{Comparison with null model}
 
Species number and functional dispersion had SESs mostly lower than $0$, while functional richness had SES usually greater than $0$. Divergence and evenness were spread either side of $0$, with medians slightly negative (Figure \ref{fig:SEShists}). Significance of these scores can be seen in Table \ref{SESrestab}.

\begin{knitrout}
\definecolor{shadecolor}{rgb}{0.969, 0.969, 0.969}\color{fgcolor}\begin{figure}[h]

{\centering \includegraphics[width=\textwidth,height=0.7\textwidth]{figure/SEShists-1} 

}

\caption[Histograms of standardised effect size (SES) for each functional diversity metric, calculated from 1000 randomly simulated communities]{Histograms of standardised effect size (SES) for each functional diversity metric, calculated from 1000 randomly simulated communities}\label{fig:SEShists}
\end{figure}


\end{knitrout}


% Table created by stargazer v.5.2 by Marek Hlavac, Harvard University. E-mail: hlavac at fas.harvard.edu
% Date and time: Tue, Jun 28, 2016 - 01:23:45
\begin{table}[tb] \centering 
  \caption{Median SES score and its associated p-value for each functional diversity metric} 
  \label{SESrestab} 
\small 
\begin{tabular}{@{\extracolsep{5pt}} ccc} 
\\[-1.8ex]\hline 
\hline \\[-1.8ex] 
Metric & Median & p value \\ 
\hline \\[-1.8ex] 
Dispersion & $$-$15.5$ & \textless 0.001 \\ 
Divergence & $$-$1$ & 0.001 \\ 
Evenness & $$-$2.7$ & \textless 0.001 \\ 
No.Species & $$-$8.2$ & \textless 0.001 \\ 
Richness & $0.4$ & \textless 0.001 \\ 
\hline \\[-1.8ex] 
\end{tabular} 
\end{table} 


% \clearpage
\subsection{Species' habitat associations}

Many species showed a clear preference for, or avoidance of, one habitat (Table \ref{hab_com_tab}), with many species appearing on lake surveys but avoiding the beach, and others appearing to be coastal habitat specialists.


% Table created by stargazer v.5.2 by Marek Hlavac, Harvard University. E-mail: hlavac at fas.harvard.edu
% Date and time: Tue, Jun 28, 2016 - 01:23:45
% Requires LaTeX packages: dcolumn 
\begin{table}[!htbp] \centering 
  \caption{Habitat associations of migratory and resident birds: percentage occurrence and mean abundance in 106 surveys} 
  \label{hab_com_tab} 
\footnotesize 
\begin{tabular}{@{\extracolsep{5pt}} D{.}{.}{-3} D{.}{.}{-3} D{.}{.}{-3} D{.}{.}{-3} D{.}{.}{-3} D{.}{.}{-3} D{.}{.}{-3} } 
\\[-1.8ex]\hline 
\hline \\[-1.8ex] 
\multicolumn{1}{c}{Species} & \multicolumn{1}{c}{Life history} & \multicolumn{1}{c}{ } & \multicolumn{1}{c}{Frequency} & \multicolumn{1}{c}{  } & \multicolumn{1}{c}{   } & \multicolumn{1}{c}{Mean} \\ 
\hline \\[-1.8ex] 
\multicolumn{1}{c}{} & \multicolumn{1}{c}{} & \multicolumn{1}{c}{Beach} & \multicolumn{1}{c}{Lake} & \multicolumn{1}{c}{     } & \multicolumn{1}{c}{Beach} & \multicolumn{1}{c}{Lake} \\ 
\multicolumn{1}{c}{Amazonetta brasiliensis} & \multicolumn{1}{c}{Resident} & \multicolumn{1}{c}{3.7} & \multicolumn{1}{c}{80.5} & \multicolumn{1}{c}{     } & \multicolumn{1}{c}{ 0.1} & \multicolumn{1}{c}{12.3} \\ 
\multicolumn{1}{c}{Anas flavirostris} & \multicolumn{1}{c}{Resident} & \multicolumn{1}{c}{0} & \multicolumn{1}{c}{21.8} & \multicolumn{1}{c}{     } & \multicolumn{1}{c}{0.0} & \multicolumn{1}{c}{0.9} \\ 
\multicolumn{1}{c}{Anas georgica} & \multicolumn{1}{c}{Resident} & \multicolumn{1}{c}{0} & \multicolumn{1}{c}{25.3} & \multicolumn{1}{c}{     } & \multicolumn{1}{c}{0.0} & \multicolumn{1}{c}{3.4} \\ 
\multicolumn{1}{c}{Anas versicolor} & \multicolumn{1}{c}{Resident} & \multicolumn{1}{c}{0} & \multicolumn{1}{c}{48.3} & \multicolumn{1}{c}{     } & \multicolumn{1}{c}{0.0} & \multicolumn{1}{c}{3.7} \\ 
\multicolumn{1}{c}{Aramus guarauna} & \multicolumn{1}{c}{Resident} & \multicolumn{1}{c}{0} & \multicolumn{1}{c}{47.1} & \multicolumn{1}{c}{     } & \multicolumn{1}{c}{0.0} & \multicolumn{1}{c}{1.1} \\ 
\multicolumn{1}{c}{Ardea alba} & \multicolumn{1}{c}{Resident} & \multicolumn{1}{c}{11.1} & \multicolumn{1}{c}{73.6} & \multicolumn{1}{c}{     } & \multicolumn{1}{c}{0.8} & \multicolumn{1}{c}{4.8} \\ 
\multicolumn{1}{c}{Ardea cocoi} & \multicolumn{1}{c}{Resident} & \multicolumn{1}{c}{74.1} & \multicolumn{1}{c}{64.4} & \multicolumn{1}{c}{     } & \multicolumn{1}{c}{1.8} & \multicolumn{1}{c}{1.5} \\ 
\multicolumn{1}{c}{Bubulcus ibis} & \multicolumn{1}{c}{Resident} & \multicolumn{1}{c}{0} & \multicolumn{1}{c}{25.3} & \multicolumn{1}{c}{     } & \multicolumn{1}{c}{0.0} & \multicolumn{1}{c}{1.9} \\ 
\multicolumn{1}{c}{Calidris alba} & \multicolumn{1}{c}{Boreal} & \multicolumn{1}{c}{63} & \multicolumn{1}{c}{2.3} & \multicolumn{1}{c}{     } & \multicolumn{1}{c}{377.0} & \multicolumn{1}{c}{  3.5} \\ 
\multicolumn{1}{c}{Calidris canutus} & \multicolumn{1}{c}{Boreal} & \multicolumn{1}{c}{29.6} & \multicolumn{1}{c}{4.6} & \multicolumn{1}{c}{     } & \multicolumn{1}{c}{85.2} & \multicolumn{1}{c}{ 2.8} \\ 
\multicolumn{1}{c}{Calidris fuscicollis} & \multicolumn{1}{c}{Boreal} & \multicolumn{1}{c}{37} & \multicolumn{1}{c}{4.6} & \multicolumn{1}{c}{     } & \multicolumn{1}{c}{177.6} & \multicolumn{1}{c}{  7.6} \\ 
\multicolumn{1}{c}{Charadrius collaris} & \multicolumn{1}{c}{Resident} & \multicolumn{1}{c}{63} & \multicolumn{1}{c}{6.9} & \multicolumn{1}{c}{     } & \multicolumn{1}{c}{8.5} & \multicolumn{1}{c}{0.4} \\ 
\multicolumn{1}{c}{Charadrius falklandicus} & \multicolumn{1}{c}{Austral} & \multicolumn{1}{c}{22.2} & \multicolumn{1}{c}{0} & \multicolumn{1}{c}{     } & \multicolumn{1}{c}{4.7} & \multicolumn{1}{c}{0.0} \\ 
\multicolumn{1}{c}{Charadrius semipalmatus} & \multicolumn{1}{c}{Boreal} & \multicolumn{1}{c}{48.1} & \multicolumn{1}{c}{2.3} & \multicolumn{1}{c}{     } & \multicolumn{1}{c}{12.2} & \multicolumn{1}{c}{ 0.4} \\ 
\multicolumn{1}{c}{Chauna torquata} & \multicolumn{1}{c}{Resident} & \multicolumn{1}{c}{3.7} & \multicolumn{1}{c}{47.1} & \multicolumn{1}{c}{     } & \multicolumn{1}{c}{0.1} & \multicolumn{1}{c}{2.2} \\ 
\multicolumn{1}{c}{Chroicocephalus maculipennis} & \multicolumn{1}{c}{Resident} & \multicolumn{1}{c}{77.8} & \multicolumn{1}{c}{50.6} & \multicolumn{1}{c}{     } & \multicolumn{1}{c}{90.9} & \multicolumn{1}{c}{15.4} \\ 
\multicolumn{1}{c}{Ciconia maguari} & \multicolumn{1}{c}{Resident} & \multicolumn{1}{c}{3.7} & \multicolumn{1}{c}{43.7} & \multicolumn{1}{c}{     } & \multicolumn{1}{c}{0.0} & \multicolumn{1}{c}{1.3} \\ 
\multicolumn{1}{c}{Coscoroba coscoroba} & \multicolumn{1}{c}{Resident} & \multicolumn{1}{c}{14.8} & \multicolumn{1}{c}{35.6} & \multicolumn{1}{c}{     } & \multicolumn{1}{c}{39.8} & \multicolumn{1}{c}{ 8.0} \\ 
\multicolumn{1}{c}{Cygnus melanocoryphus} & \multicolumn{1}{c}{Resident} & \multicolumn{1}{c}{0} & \multicolumn{1}{c}{10.3} & \multicolumn{1}{c}{     } & \multicolumn{1}{c}{0.0} & \multicolumn{1}{c}{2.3} \\ 
\multicolumn{1}{c}{Dendrocygna bicolor} & \multicolumn{1}{c}{Resident} & \multicolumn{1}{c}{0} & \multicolumn{1}{c}{24.1} & \multicolumn{1}{c}{     } & \multicolumn{1}{c}{ 0.0} & \multicolumn{1}{c}{46.2} \\ 
\multicolumn{1}{c}{Dendrocygna viduata} & \multicolumn{1}{c}{Resident} & \multicolumn{1}{c}{0} & \multicolumn{1}{c}{44.8} & \multicolumn{1}{c}{     } & \multicolumn{1}{c}{ 0.0} & \multicolumn{1}{c}{90.8} \\ 
\multicolumn{1}{c}{Egretta thula} & \multicolumn{1}{c}{Resident} & \multicolumn{1}{c}{44.4} & \multicolumn{1}{c}{58.6} & \multicolumn{1}{c}{     } & \multicolumn{1}{c}{33.1} & \multicolumn{1}{c}{ 3.5} \\ 
\multicolumn{1}{c}{Fulica leucoptera} & \multicolumn{1}{c}{Resident} & \multicolumn{1}{c}{0} & \multicolumn{1}{c}{25.3} & \multicolumn{1}{c}{     } & \multicolumn{1}{c}{ 0.0} & \multicolumn{1}{c}{33.7} \\ 
\multicolumn{1}{c}{Gallinula chloropus} & \multicolumn{1}{c}{Resident} & \multicolumn{1}{c}{0} & \multicolumn{1}{c}{47.1} & \multicolumn{1}{c}{     } & \multicolumn{1}{c}{ 0.0} & \multicolumn{1}{c}{33.6} \\ 
\multicolumn{1}{c}{Haematopus palliatus} & \multicolumn{1}{c}{Resident} & \multicolumn{1}{c}{92.6} & \multicolumn{1}{c}{6.9} & \multicolumn{1}{c}{     } & \multicolumn{1}{c}{194.9} & \multicolumn{1}{c}{  0.3} \\ 
\multicolumn{1}{c}{Himantopus mexicanus} & \multicolumn{1}{c}{Resident} & \multicolumn{1}{c}{40.7} & \multicolumn{1}{c}{42.5} & \multicolumn{1}{c}{     } & \multicolumn{1}{c}{11.9} & \multicolumn{1}{c}{ 8.8} \\ 
\multicolumn{1}{c}{Jacana jacana} & \multicolumn{1}{c}{Resident} & \multicolumn{1}{c}{0} & \multicolumn{1}{c}{56.3} & \multicolumn{1}{c}{     } & \multicolumn{1}{c}{0.0} & \multicolumn{1}{c}{4.6} \\ 
\multicolumn{1}{c}{Larus dominicanus} & \multicolumn{1}{c}{Resident} & \multicolumn{1}{c}{85.2} & \multicolumn{1}{c}{14.9} & \multicolumn{1}{c}{     } & \multicolumn{1}{c}{236.6} & \multicolumn{1}{c}{  2.7} \\ 
\multicolumn{1}{c}{Mycteria americana} & \multicolumn{1}{c}{Resident} & \multicolumn{1}{c}{0} & \multicolumn{1}{c}{10.3} & \multicolumn{1}{c}{     } & \multicolumn{1}{c}{0.0} & \multicolumn{1}{c}{0.6} \\ 
\multicolumn{1}{c}{Netta peposaca} & \multicolumn{1}{c}{Resident} & \multicolumn{1}{c}{0} & \multicolumn{1}{c}{12.6} & \multicolumn{1}{c}{     } & \multicolumn{1}{c}{ 0} & \multicolumn{1}{c}{48} \\ 
\multicolumn{1}{c}{Nycticorax nycticorax} & \multicolumn{1}{c}{Resident} & \multicolumn{1}{c}{0} & \multicolumn{1}{c}{14.9} & \multicolumn{1}{c}{     } & \multicolumn{1}{c}{0.0} & \multicolumn{1}{c}{0.6} \\ 
\multicolumn{1}{c}{Phaetusa simplex} & \multicolumn{1}{c}{Resident} & \multicolumn{1}{c}{22.2} & \multicolumn{1}{c}{3.4} & \multicolumn{1}{c}{     } & \multicolumn{1}{c}{0.7} & \multicolumn{1}{c}{0.1} \\ 
\multicolumn{1}{c}{Phalacrocorax brasilianus} & \multicolumn{1}{c}{Resident} & \multicolumn{1}{c}{48.1} & \multicolumn{1}{c}{55.2} & \multicolumn{1}{c}{     } & \multicolumn{1}{c}{10.9} & \multicolumn{1}{c}{52.6} \\ 
\multicolumn{1}{c}{Phimosus infuscatus} & \multicolumn{1}{c}{Resident} & \multicolumn{1}{c}{0} & \multicolumn{1}{c}{51.7} & \multicolumn{1}{c}{     } & \multicolumn{1}{c}{0.0} & \multicolumn{1}{c}{4.4} \\ 
\multicolumn{1}{c}{Phoenicopterus chilensis} & \multicolumn{1}{c}{Austral} & \multicolumn{1}{c}{14.8} & \multicolumn{1}{c}{12.6} & \multicolumn{1}{c}{     } & \multicolumn{1}{c}{53.3} & \multicolumn{1}{c}{ 6.0} \\ 
\multicolumn{1}{c}{Platalea ajaja} & \multicolumn{1}{c}{Resident} & \multicolumn{1}{c}{0} & \multicolumn{1}{c}{28.7} & \multicolumn{1}{c}{     } & \multicolumn{1}{c}{0.0} & \multicolumn{1}{c}{2.3} \\ 
\multicolumn{1}{c}{Plegadis chihi} & \multicolumn{1}{c}{Resident} & \multicolumn{1}{c}{3.7} & \multicolumn{1}{c}{78.2} & \multicolumn{1}{c}{     } & \multicolumn{1}{c}{  0.2} & \multicolumn{1}{c}{108.1} \\ 
\multicolumn{1}{c}{Pluvialis dominica} & \multicolumn{1}{c}{Boreal} & \multicolumn{1}{c}{25.9} & \multicolumn{1}{c}{3.4} & \multicolumn{1}{c}{     } & \multicolumn{1}{c}{5.3} & \multicolumn{1}{c}{0.6} \\ 
\multicolumn{1}{c}{Podiceps major} & \multicolumn{1}{c}{Resident} & \multicolumn{1}{c}{0} & \multicolumn{1}{c}{26.4} & \multicolumn{1}{c}{     } & \multicolumn{1}{c}{0.0} & \multicolumn{1}{c}{0.5} \\ 
\multicolumn{1}{c}{Podilymbus podiceps} & \multicolumn{1}{c}{Resident} & \multicolumn{1}{c}{0} & \multicolumn{1}{c}{17.2} & \multicolumn{1}{c}{     } & \multicolumn{1}{c}{0.0} & \multicolumn{1}{c}{0.3} \\ 
\multicolumn{1}{c}{Rollandia rolland} & \multicolumn{1}{c}{Resident} & \multicolumn{1}{c}{0} & \multicolumn{1}{c}{12.6} & \multicolumn{1}{c}{     } & \multicolumn{1}{c}{0.0} & \multicolumn{1}{c}{0.2} \\ 
\multicolumn{1}{c}{Rynchops niger} & \multicolumn{1}{c}{Resident} & \multicolumn{1}{c}{48.1} & \multicolumn{1}{c}{8} & \multicolumn{1}{c}{     } & \multicolumn{1}{c}{42.7} & \multicolumn{1}{c}{13.9} \\ 
\multicolumn{1}{c}{Sterna hirundinacea} & \multicolumn{1}{c}{Boreal} & \multicolumn{1}{c}{29.6} & \multicolumn{1}{c}{0} & \multicolumn{1}{c}{     } & \multicolumn{1}{c}{143.7} & \multicolumn{1}{c}{  0.0} \\ 
\multicolumn{1}{c}{Sterna hirundo} & \multicolumn{1}{c}{Boreal} & \multicolumn{1}{c}{44.4} & \multicolumn{1}{c}{3.4} & \multicolumn{1}{c}{     } & \multicolumn{1}{c}{113.9} & \multicolumn{1}{c}{  0.1} \\ 
\multicolumn{1}{c}{Sterna sandvicensis} & \multicolumn{1}{c}{Austral} & \multicolumn{1}{c}{33.3} & \multicolumn{1}{c}{0} & \multicolumn{1}{c}{     } & \multicolumn{1}{c}{8.3} & \multicolumn{1}{c}{0.0} \\ 
\multicolumn{1}{c}{Sterna trudeaui} & \multicolumn{1}{c}{Resident} & \multicolumn{1}{c}{55.6} & \multicolumn{1}{c}{3.4} & \multicolumn{1}{c}{     } & \multicolumn{1}{c}{21.4} & \multicolumn{1}{c}{ 0.2} \\ 
\multicolumn{1}{c}{Sternula superciliaris} & \multicolumn{1}{c}{Resident} & \multicolumn{1}{c}{51.9} & \multicolumn{1}{c}{14.9} & \multicolumn{1}{c}{     } & \multicolumn{1}{c}{3.1} & \multicolumn{1}{c}{1.0} \\ 
\multicolumn{1}{c}{Syrigma sibilatrix} & \multicolumn{1}{c}{Resident} & \multicolumn{1}{c}{3.7} & \multicolumn{1}{c}{5.7} & \multicolumn{1}{c}{     } & \multicolumn{1}{c}{0.1} & \multicolumn{1}{c}{0.1} \\ 
\multicolumn{1}{c}{Thalasseus maximus} & \multicolumn{1}{c}{Resident} & \multicolumn{1}{c}{40.7} & \multicolumn{1}{c}{1.1} & \multicolumn{1}{c}{     } & \multicolumn{1}{c}{5.9} & \multicolumn{1}{c}{0.0} \\ 
\multicolumn{1}{c}{Tringa flavipes} & \multicolumn{1}{c}{Boreal} & \multicolumn{1}{c}{11.1} & \multicolumn{1}{c}{10.3} & \multicolumn{1}{c}{     } & \multicolumn{1}{c}{2.9} & \multicolumn{1}{c}{2.4} \\ 
\multicolumn{1}{c}{Tringa melanoleuca} & \multicolumn{1}{c}{Boreal} & \multicolumn{1}{c}{22.2} & \multicolumn{1}{c}{8} & \multicolumn{1}{c}{     } & \multicolumn{1}{c}{1.6} & \multicolumn{1}{c}{0.6} \\ 
\multicolumn{1}{c}{Vanellus chilensis} & \multicolumn{1}{c}{Resident} & \multicolumn{1}{c}{14.8} & \multicolumn{1}{c}{16.1} & \multicolumn{1}{c}{     } & \multicolumn{1}{c}{1.0} & \multicolumn{1}{c}{2.2} \\ 
\hline \\[-1.8ex] 
\end{tabular} 
\end{table} 



\end{document}


\section{Discussion}

\printbibliography


\end{document}